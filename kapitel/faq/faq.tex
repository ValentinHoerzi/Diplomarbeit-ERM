\chapter{FAQ}
\section{Was ist FAQ}
FAQ steht für "Frequently Asked Questions". Es handelt sich dabei um eine Ansammlung von häufig gestellten Fragen, die auch für weitere Benutzer interessant sein könnten. Somit findet man auf einer Webseite im Bereich FAQ eine Auflistung dieser Ansammlung von wiederkehrenden Benutzerfragen.
\section{FAQ-Backend}
Eine der Aufgaben war es, ein funktionierendes Backend für einen zukünftigen FAQ-Bereich zu programmieren. Wenn eine Person eine Frage gestellt bekommt und diese bearbeitet, beantwortet und dabei empfindet, dass diese Frage auch noch für weitere Benutzer von Interesse sein könnte und diese Person auch die erforderlichen Rechte dafür besitzt, kann sie aus dieser Frage eine FAQ erstellen.\\
Dafür wurde zuerst eine Tabelle in der Datenbank definiert. Ein FAQ besteht in der Datenbank aus einer ID, einem Sprachenkürzel, dem Titel, den Inhalt der Frage, einem Zähler, der die Abfragen mitzählt und einer Liste mit den vorhandenen Übersetzungen für diese Frage. Die Berechtigung ist mit einem Claim in dem JWT gespeichert. Wenn nun der Befehl um eine FAQ zu erstellen aufgerufen wird, wird zuerst überprüft, ob die benötigte Berechtigung vorhanden ist. Falls jemand berechtigt ist, wird die FAQ erstellt und automatisch mit Google Cloud Translate übersetzt und gespeichert. In diesen folgenden Sprachen ist die FAQ danach übersetzt:
\begin{multicols}{2}
	\begin{itemize}
		\item Deutsch
		\item Englisch
		\item Französisch
		\item Niederländisch
		\item Dänisch
		\item Spanisch
		\item Italienisch
		\item Ungarisch
		\item Polnisch
		\item Slowakisch
		\item Finnisch
		\item Schwedisch
		\item Tschechisch
		\item Russisch
		\item Ukrainisch
	\end{itemize}
\end{multicols}