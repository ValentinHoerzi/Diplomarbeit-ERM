\chapter{Command-query Separation}
Das Prinzip der CQS(Command-query Separation) steht für die Trennung von Befehlen und Abfragen. In der klassischen Geschäftslogik-Schicht gibt es mehrere Schnittstellen, die verschiedenste Aufgaben abdecken. Da die Möglichkeit besteht, dass  mit der Zeit immer mehr Anforderungen an das System gestellt werden, werden die Schnittstellen und die sich darin befindenden Methoden immer mehr und möglicherweise auch die Cross-Cutting Concerns. Mit Cross-Cutting Concerns sind Anforderungen gemeint, die das ganze System betreffen, aber nicht vom Anwendungsfall abhängig sind. Das können Transaktionen, Validierungen oder Rechteüberprüfungen sein. In dieser Diplomarbeit wurde bei jedem Aufruf geprüft, ob der Benutzer die benötigten Rechte besitzt. Somit wäre ohne CQS in jeder Methode etwas zu ändern. \autocite{cqsSOLIDeArchitektur}\\
\section{Unterschied zwischen Abfragen und Befehlen}
Abfragen sollten in diesem Fall idempotent sein. Das bedeutet, wenn die selbe Abfrage mehrmals nacheinander ausgeführt wird, jedes mal dasselbe Ergebnis zurück kommen sollte. Somit sollen Abfragen nur Ergebnisse liefern und keine Daten verändern.
Befehle sind nur zur Datenveränderung vorhanden und sind daher nicht für Datenabfragen vorgesehen. Befehle sind daher nicht idempotent, da jeder Aufruf Datensets verändert.
\section{Umsetzung}
Um dies umzusetzen, werden die zwei generische Schnittstellen benötigt, die man im Listing \ref{lst:cqsSchnittstellen} sieht. \texttt{IQuery} steht für Abfragen und \texttt{ICommand} für Befehle.
\begin{lstlisting}[caption={CQS-Schnittstellen},captionpos=b, numbers=left, backgroundcolor=\color{black!10},language={[Sharp]C}, label={lst:cqsSchnittstellen}]
	public interface IQuery<TResult> { }
	public interface ICommand { }
\end{lstlisting}
Eine Methode aus dem Controller wird nun eine eigene Klasse. Als Eigenschaften dieser neuen Klasse werden die Methodenparameter verwendet. Ein Beispiel aus dieser Diplomarbeit kann man im Listing \ref{lst:getFaqbyid} sehen. Es implementiert das im Listing \ref{lst:cqsSchnittstellen} vorhandene Interface \texttt{IQuery} mit dem Datentyp FAQDto.

\newpage

\begin{lstlisting}[caption={CQS-Query Beispiel},captionpos=b, numbers=left, backgroundcolor=\color{black!10},language={[Sharp]C}, label={lst:getFaqbyid}]
	public class GetFAQById : IQuery<FAQDto>
	{
		public GetFAQById(Guid faqId)
		{
			this.FAQId = faqId;
		}
		public Guid FAQId { get;}
	}
\end{lstlisting}
Um die gewollte Logik auszuführen, werden zwei weitere generischen Schnittstellen benötigt. Wiederum gibt es eine für Abfragen und eine für Befehle, wie man im Listing \ref{lst:cqsHandler} sehen kann. Diese werden als Handler betitelt. \autocite{cqsSOLIDeArchitektur}
\begin{lstlisting}[caption={CQS-Handler},captionpos=b, numbers=left, backgroundcolor=\color{black!10},language={[Sharp]C}, label={lst:cqsHandler}]
	public interface IQueryHandler<TQuery, TResult>	where TQuery : IQuery<TResult>
	{
		Task<TResult> HandleAsync(TQuery query);
	}
	
	public interface ICommandHandler<TCommand> where TCommand : ICommand
	{
		Task HandleAsync(TCommand command);
	}
\end{lstlisting}
Den Handler für die im Listing \ref{lst:getFaqbyid} zu sehende Abfrage sieht man im Listing \ref{getFaqByIdHandler}.
\begin{lstlisting}[caption={CQS-Handler Beispiel},captionpos=b, numbers=left, backgroundcolor=\color{black!10},language={[Sharp]C}, label={getFaqByIdHandler}]
	internal class GetFAQByIdHandler : IQueryHandler<GetFAQById, FAQDto>
	{
		private readonly FAQRepositoryFactory faqRepositoryFactory;
		private readonly IMapper mapper;
		
		public GetFAQByIdHandler(
		FAQRepositoryFactory faqRepositoryFactory,
		IMapper mapper)
		{
			this.faqRepositoryFactory = faqRepositoryFactory ?? throw new ArgumentNullException (nameof(faqRepositoryFactory));
			this.mapper = mapper ?? throw new ArgumentNullException(nameof(mapper));
		}
		
		public async Task<FAQDto> HandleAsync(GetFAQById query)
		{
			using(var repository = this.faqRepositoryFactory())
			{
				var faq = (await repository.GetAsync(f => f.Id == query.FAQId))
				.SingleOrDefault()
				.ThrowIfNotFound(query.FAQId);
				faq.Fetches++;
				await repository.UpsertOneAsync(f => f.Id == faq.Id, faq);
				return this.mapper.Map<FAQDto>(faq);
			}
		}
	}
\end{lstlisting}
In der Methode HandleAsync im Listing \ref{getFaqByIdHandler} wird die geforderte FAQ geholt, bei dieser danach die Anzahl der Abfragen um eins erhöht und wieder gespeichert. Danach wird diese FAQ als FAQDto zurückgegeben.\\
Dependency Injection wird genutzt, um die Queries mit den richtigen Query-Handlers und die Commands mit den richtigen Command-Handlers zusammenzuhängen. \autocite{cqsSOLIDeArchitektur}

\newpage

\section{Dependency Injection}
In der objektorientierten Programmierung gilt DI (Dependency Injection) als Entwurfsmuster. Dabei werden die Abhängigkeit von den Objekten erst zur Laufzeit geregelt. Falls bei der Initialisierung eines Objektes ein anderes Objekt gebraucht wird, ist das an einem zentralen Ort hinterlegt. Dadurch wird die Verantwortlichkeit für die Abhängigkeiten der Klassen zentral gesammelt.\autocite{wikiDI}\\
Durch DI hat man einige Vorteile. Einer davon, ist die Flexibilität, die der Client dadurch erhält, somit kann dieser auf alles reagieren, was die Schnittstelle bietet. Weiters können die Konfigurationsdetails in Konfigurationsdateien verlegt werden. Dadurch benötigt es nach einer Neukonfiguration keine Neukompilierung. Indem das gesamte Implementierungswissen ausgelagert wird, ist die Wartbarkeit, Testbarkeit und die Wiederverwendbarkeit gefördert. Es ist möglich, dass gleichzeitig zwei Klassen von unterschiedlichen Entwicklern programmiert werden, die voneinander abhängig sind, da sie nur die Schnittstellen dafür kennen müssen.\autocite{wikiDI}\\
Ein Nachteil, der durch DI entsteht, ist, dass die Lesbarkeit geringer wird, da man um etwas zu verstehen, in weiteren Dateien nachsehen muss. Da DI-Frameworks mit Reflexion oder dynamischer Programmierung implementiert werden, kann die Leistung des Systems gesenkt werden. Es ist auch teilweise mehr Aufwand nötig, da man es nicht voraussagen kann, wo es benötigt wird. \autocite{wikiDI}
