\chapter{Command-query Separation}
Das Prinzip der Command-query Separation (CQS) steht für die Trennung von Befehlen und Abfragen. In der klassischen Geschäftlogik-Schicht gibt es mehrere Schnittstellen, die verschiedenste Aufgaben abdecken. Da die Möglichkeit besteht, dass  mit der Zeit immer mehr Anforderungen an das System gestellt werden, werden die Schnittstellen und die darin befindenden Methoden immer mehr und möglicherweise auch die Cross-Cutting Concerns. Mit Cross-Cutting Concerns sind Anforderungen gemeint, die das ganze System betreffen, aber nicht von Anwendungsfall abhängig sind. Das können Transaktionen, Validierungen oder Rechteüberprüfungen sein. In dieser Diplomarbeit wurde bei jeden Aufruf geprüft, ob der Benutzer die berechtigten Rechte hat. Somit wäre ohne CQS in jeder Methode etwas zu ändern. \autocite{cqsSOLIDeArchitektur}\\
\section{Unterschied zwischen Abfragen und Befehlen}
Abfragen sollten in diesem Fall idempotent sein. Das bedeutet, dass wenn die selbe Abfrage öfters nacheinander ausgeführt wird, jedes mal dasselbe Ergebnis zurück kommen sollte. Somit sollen Abfragen nur Ergebnisse liefern und keine Daten verändern.
Befehle sind nur zur Datenveränderung vorhanden und sind daher nicht für Datenabfragen da. Befehle sind daher nicht idempotent, da jeder Aufruf Datensets verändert.
\section{Umsetzung}
Um das umzusetzen, werden die zwei generische Schnittstellen benötigt, die man in Listing \ref{lst:cqsSchnittstellen} sieht. \texttt{IQuery} steht für Abfragen und \texttt{ICommand}, die für Befehlen steht.
\begin{lstlisting}[caption={CQS-Schnittstellen},captionpos=b, numbers=left, backgroundcolor=\color{black!10},language={[Sharp]C}, label={lst:cqsSchnittstellen}]
	public interface IQuery<TResult> { }
	public interface ICommand { }
\end{lstlisting}
Eine Methode aus dem Controller wird nun eine eigene Klasse. Als Eigenschaften dieser neuen Klasse werden die Methodenparameter verwendet. Ein Beispiel aus dieser Diplomarbeit kann man im Listing \ref{lst:getFaqbyid} sehen. Es implementiert das im Listing \ref{lst:cqsSchnittstellen} vorhandene Interface \texttt{IQuery} mit dem Datentyp FAQDto.
\begin{lstlisting}[caption={CQS-Query Beispiel},captionpos=b, numbers=left, backgroundcolor=\color{black!10},language={[Sharp]C}, label={lst:getFaqbyid}]
	public class GetFAQById : IQuery<FAQDto>
	{
		public GetFAQById(Guid faqId)
		{
			this.FAQId = faqId;
		}
		public Guid FAQId { get;}
	}
\end{lstlisting}
Um die gewollte Logik auszuführen, werden zwei weitere generischen Schnittstellen benötigt. Wiederum gitb es eine für Abfragen und eine für Befehle, wie man im Listing \ref{lst:cqsHandler} sehen kann. Diese werden als Handler betitelt. \autocite{cqsSOLIDeArchitektur}
\begin{lstlisting}[caption={CQS-Handler},captionpos=b, numbers=left, backgroundcolor=\color{black!10},language={[Sharp]C}, label={lst:cqsHandler}]
	public interface IQueryHandler<TQuery, TResult>	where TQuery : IQuery<TResult>
	{
		Task<TResult> HandleAsync(TQuery query);
	}
	
	public interface ICommandHandler<TCommand> where TCommand : ICommand
	{
		Task HandleAsync(TCommand command);
	}
\end{lstlisting}
Den Handler für die im Listing \ref{lst:getFaqbyid} zu sehende Abfrage sieht man im Listing \ref{getFaqByIdHandler}.
\begin{lstlisting}[caption={CQS-Handler Beispiel},captionpos=b, numbers=left, backgroundcolor=\color{black!10},language={[Sharp]C}, label={getFaqByIdHandler}]
	internal class GetFAQByIdHandler : IQueryHandler<GetFAQById, FAQDto>
	{
		private readonly FAQRepositoryFactory faqRepositoryFactory;
		private readonly IMapper mapper;
		
		public GetFAQByIdHandler(
		FAQRepositoryFactory faqRepositoryFactory,
		IMapper mapper)
		{
			this.faqRepositoryFactory = faqRepositoryFactory ?? throw new ArgumentNullException (nameof(faqRepositoryFactory));
			this.mapper = mapper ?? throw new ArgumentNullException(nameof(mapper));
		}
		
		public async Task<FAQDto> HandleAsync(GetFAQById query)
		{
			using(var repository = this.faqRepositoryFactory())
			{
				var faq = (await repository.GetAsync(f => f.Id == query.FAQId))
				.SingleOrDefault()
				.ThrowIfNotFound(query.FAQId);
				faq.Fetches++;
				await repository.UpsertOneAsync(f => f.Id == faq.Id, faq);
				return this.mapper.Map<FAQDto>(faq);
			}
		}
	}
\end{lstlisting}
Dependency Injection wird genutzt, um die Queries mit den richtigen Query-Handlers und die Commands mit den richtigen Command-Handlers zusammen zu hängen. \autocite{cqsSOLIDeArchitektur}
\section{Dependency Injection}
Bei DI (Dependency Injection) erhalten Klassen oder Objekte ihre Zuweisungen über spezielle Methoden von außen. 