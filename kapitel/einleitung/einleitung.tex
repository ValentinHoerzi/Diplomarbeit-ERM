\chapter{Einleitung} \label{sec:einleitung}
\section{Projektteam}
Das Projektteam besteht aus zwei Entwicklern unter denen die Arbeit gleichmäßig verteilt ist.
\subsection{Alexander Schörgendorfer}
Alexander Schörgendorfer, geboren am 04. Oktober 2001 als Sohn von Franz und Renate Schörgendorfer, wohnt in Schlüßlberg in Oberösterreich. Er besuchte von 2008 bis 2012 die Volksschule in Schlüßlberg und von 2012 bis 2016 die damalige Hauptschule 2 in Grieskirchen. Seit September 2016 besucht er die Höhere Technische Bundeslehranstalt in Grieskirchen, Fachrichtung Informatik, und befindet sich derzeit im Maturajahrgang. Alexander Schörgendorfer war besonders für das Backend und den externen Services der Diplomarbeit zuständig.\\
Die Aufgaben von Alexander Schörgendorfer sind eine FAQ Schnittstelle im Backend mit speziellen Berechtigungen, Adressenvalidierung im Backend, Telefonnummernvalidierung im Backend, Login / Logout mit Auth0, sämtliche Pflichtfeldvalidierung im Backend, Suche nach Services, die die benötigten Daten liefern und diese verschiedenen Services vergleichen, Beschaffung der benötigten Daten für den dynamischen Aufbau des Registrierungsformulars von diversen zuvor recherchierten Services. Die gesamte Arbeit soll in DSGVO konformer Form und unter Berücksichtigung des BSI IT-Grundschutzkompendiums vollzogen werden. Das Design des Frontends erfolgt nach der Designvorlage, die von Pöttinger Landtechnik GmbH erstellt worden ist.
\subsection{Valentin Hörzi}
Valentin Hörzi, geboren am 31. Oktober 2001 als Sohn von Christian Strassl und Monika Hörzi, wohnt in Wels in Oberösterreich. Er besuchte von 2008 bis 2011 das Integrative Schulzentrum in Wels bis er schließlich sein letztes Jahr an der Volksschule von 2011 bis 2012 an der Volksschule 3 in der Doktor Schauer-Straße in Wels absolvierte. Von 2012 bis 2016 schloss er die Unterstufe in dem Bundesgymnasium und Bundesrealgymnasium Wels, ebenfalls in der Doktor Schauer-Straße, ab. Seit September 2016 besucht er die Höhere Technische  Bundeslehranstalt in Grieskirchen, Fachrichtung Informatik und befindet sich derzeit im Maturajahrgang. Das Hauptaugenmerk von Valentin galt bei der Entwicklung der Diplomarbeit vor allem dem Algorithmus, welcher den Aufbau des Registrierungsformulars steuerte.
Dieser Algorithmus gewährleistet, dass sich das Registrierungsformular sowie die Profilübersicht, dynamisch an das jeweilige ausgewählte Land und der derzeitigen Benutzergruppe anpasst. So entsteht die Möglichkeit für amerikanische Benutzer zusätzlich ihren Bundesstaat auswählen zu können, beziehungsweise italienische Kunden ihre zugehörige Provinz. Dazu wurde auch eine Validierungsmethodik, welche ebenfalls dynamisch auf neue Anforderungen für das Registrierungsformular reagiert, von Valentin entwickelt. Um die Registrierung für den Benutzer so leicht wie möglich zu machen wurden von Valentin auch mehrere Komfortfunktionen eingebaut: automatische Adressvervollständigungsvorschläge, die Beschaffung der Daten des Standorts durch die IP, welche unmittelbar in das Registrierungsformular eingefügt und validiert werden und eine automatische Telefonnummernpräfix-Erkennung.
Auch sorgte Valentin durch eine komplett überarbeitete GitLab Pipeine für eine erhebliche Verbesserung des CI/CD-Geschehens.