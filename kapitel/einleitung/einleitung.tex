\chapter{Einleitung} \label{sec:einleitung}
Aus Gründen der besseren Lesbarkeit wird in dieser Diplomarbeit die Sprachform des generischen Maskulinums angewandt. An dieser Stelle weisen wir darauf hin, dass die ausschließliche Verwendung der männlichen Form geschlechtsunabhängig verstanden werden soll.
\section{Projektteam}
Das Projektteam besteht aus zwei Entwicklern unter denen die Arbeit gleichmäßig verteilt wurde.
\subsection{Alexander Schörgendorfer}
Alexander Schörgendorfer, geboren am 04. Oktober 2001 als Sohn von Franz und Renate Schörgendorfer, wohnt in Schlüßlberg, Oberösterreich. Er besuchte von 2008 bis 2012 die Volksschule in Schlüßlberg und von 2012 bis 2016 die damalige Hauptschule 2 in Grieskirchen. Seit September 2016 besucht er die Höhere Technische Bundeslehranstalt in Grieskirchen, Fachrichtung Informatik, und befindet sich derzeit im Maturajahrgang. Alexander Schörgendorfer war besonders für das Backend und die externen Services der Diplomarbeit zuständig.\\
Die Aufgaben von Alexander Schörgendorfer sind eine FAQ Schnittstelle im Backend mit speziellen Berechtigungen, Adressenvalidierung im Backend, Telefonnummernvalidierung im Backend, Login/Logout mit Auth0, sämtliche Pflichtfeldvalidierung im Backend, Suche nach Services, die die benötigten Daten liefern und diese verschiedenen Services vergleichen, Beschaffung der benötigten Daten für den dynamischen Aufbau des Registrierungsformulars von diversen zuvor recherchierten Services. Die gesamte Arbeit soll in DSGVO konformer Weise und unter Berücksichtigung des BSI IT-Grundschutzkompen\-diums vollzogen werden. Das Design des Frontends erfolgte nach der Designvorlage, die von PÖTTINGER Landtechnik GmbH erstellt worden ist.

\newpage

\subsection{Valentin Hörzi}
Valentin Hörzi, geboren am 31. Oktober 2001 als Sohn von Christian Strassl und Monika Hörzi, wohnt in Wels, Oberösterreich. Er besuchte von 2008 bis 2011 das Integrative Schulzentrum in Wels bis er schließlich sein letztes Jahr an der Volksschule von 2011 bis 2012 an der Volksschule 3 in der Doktor Schauer-Straße in Wels absolvierte. Von 2012 bis 2016 schloss er die Unterstufe in dem Bundesgymnasium und Bundesrealgymnasium Wels, ebenfalls in der Doktor Schauer-Straße, ab. Seit September 2016 besucht er die Höhere Technische  Bundeslehranstalt in Grieskirchen, Fachrichtung Informatik und befindet sich derzeit im Maturajahrgang. Das Hauptaugenmerk von Valentin galt bei der Entwicklung der Diplomarbeit vor Allem dem Algorithmus, welcher den Aufbau des Registrierungsformulars steuert.
Dieser Algorithmus gewährleistet, dass sich das RF, sowie die Profilübersicht, dynamisch an das jeweilige ausgewählte Land und der derzeitigen Benutzergruppe anpasst. So entsteht die Möglichkeit für amerikanische Benutzer, zusätzlich ihren Bundesstaat auswählen zu können, beziehungsweise italienische Kunden ihre zugehörige Provinz. Dazu wurde auch eine Validierungsmethodik, welche ebenfalls dynamisch auf neue Anforderungen für das RF reagiert, von Valentin entwickelt. Um die Registrierung für den Benutzer so leicht und so schnell wie möglich zu machen, wurden von Valentin auch mehrere Komfortfunktionen eingebaut: automatische Adressvervollständigungsvorschläge, die Beschaffung der Daten des Standorts durch die IP, welche unmittelbar in das RF eingefügt und validiert werden und eine automatische Telefonnummernpräfix-Erkennung.
Auch sorgte Valentin durch eine komplett überarbeitete GitLab Pipeline für eine erhebliche Verbesserung des CI/CD-Geschehens.

\section{Projektbetreuer}

Herr Professor \ThSupervisorName \, war auf Seiten der HTBLA Grieskirchen unsere Ansprechperson. Bei Fragen über das Projekt oder die Diplomarbeit an sich unterstütze er uns.

\section{Auftraggeber}

Die Diplomarbeit wurde in Kooperation mit dem Unternehmen \ThPartnerName \, durchgeführt. Schörgendorfer Alexander kam in den Sommerferien 2019 erstmals durch ein Praktikum mit diesem Unternehmen in Kontakt. Durch diesen Bezug entwickelte sich auch die Zusammenarbeit mit der Diplomarbeit. \ThPartnerName \, wurde 1871 gegründet und produziert Landmaschinen. Die Zentrale des Unternehmens befindet sich in Grieskirchen. \newpage
Mit der Entwicklung und Herstellung von innovativer Landtechnik leistet PÖTTINGER einen wertvollen Beitrag zur Effizienzsteigerung der Agrarproduktion. PÖTTINGER entwickelt Ideen, die die tägliche Arbeit der Landwirte erleichtern. Die Produktpalette umfasst Landmaschinen für die Bereiche Grünland, Bodenbearbeitung und Sätechnik.\\
PÖTTINGER beschäftigt weltweit über 1.900 Mitarbeiter/innen und erzielt einen  jährlichen Umsatz von ca. 400 Millionen Euro.

\subsection{Hubert Kerschhuber}

Herr Kerschhuber ist Leiter der Abteilung IT Entwicklung und war durch diese Position die Hauptansprechperson für die allgemeine Abwicklung der Diplomarbeit.

\subsection{Dominik Augustin}

Herr Augustin war bis Jahresende 2020 unsere Ansprechperson für die technische Abwicklung. Er unterstützte uns bei jeglichen Fragen über das Projekt und erfüllte jede offene Bitte.

\subsection{Florian Mittlböck}

Herr Mittlböck ist seit Jahresbeginn 2021 unsere Ansprechperson für die technische Abwicklung. Er übernahm von diesem Zeitpunkt an die Stelle von Herrn Augustin in dieser Diplomarbeit.
