\chapter{Internationalisierung}
In der Informatik bezeichnet man ein Programm als internationalisiert, wenn es mit dem unveränderten Quellcode in andere Sprachen und Kulturen übersetzt werden kann. Diese Bezeichnung wird oftmals als i18n abgekürzt. Die Zahl 18 steht für die Anzahl der Buchstaben, die sich zwischen dem ersten und den letzten Buchstaben im englischen Wort "internationalization" stehen. \autocite{wikii18n}
\section{Umsetzung}
Damit für eine Übersetzung der Quellcode nicht verändert werden muss, darf ein Text nicht hartkodiert sein. Für die verwendeten Texte sollen 