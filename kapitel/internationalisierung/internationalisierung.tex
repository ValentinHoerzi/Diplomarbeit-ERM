\chapter{Internationalisierung}
In der Informatik bezeichnet man ein Programm als internationalisiert, wenn es mit dem unveränderten Quellcode in andere Sprachen und Kulturen übersetzt werden kann. Diese Bezeichnung wird oftmals als i18n abgekürzt. Die Zahl 18 steht für die Anzahl der Buchstaben, die sich zwischen dem ersten und den letzten Buchstaben im englischen Wort "internationalization" stehen. \autocite{wikii18n}
\section{Umsetzung}
Damit für eine Übersetzung der Quellcode nicht verändert werden muss, darf ein Text nicht hartkodiert sein. Für die verwendeten Texte sollen Variablen genutzt werden, die sich der gewünschten Sprache zur Laufzeit anpassen. Die Variablen werden in JSON-Dateien gespeichert und in den jeweilig Verfügbaren Sprachen übersetzt. Somit ergeben sich einige JSON-Dateien, die geladen werden können. Durch spezielle Plug-Ins ist es möglich, die neuen Variablen in einer Sprache selber zu definieren und danach mit einem Klick in die gewünschten Sprachen übersetzen zu lassen. Dadurch füllen sich die weiteren JSON-Dateien.\autocite{wikii18n}\\
Im Bereich des RFs ist die Internationalisierung nicht nur für die Übersetzung integriert, sondern auch für den korrekten Aufbau und die Validierung der Felder. Dabei richtet sich die Reihenfolge der personenbezogenen Daten und der Adresseingabe und die Pflichtfelder nach dem Standard des ausgewählten Landes. Beispiele dafür sind Länder mit Provinzen oder welche, die keine Postleitzahlen besitzen. Solche Felder werden je nach Vorgabe angezeigt. Ein Beispiel für die veränderte Validierung ist die Länge der Postleitzahl, die sehr unterschiedlich sein kann.\autocite{wikii18n}
\section{Umstände}
Im Laufe der Zeit kann es sein, dass sich Texte ändern. Somit sind auch alle Übersetzungen neu zu übersetzen. Da nun der neu definierte Text möglicherweise länger oder kürzer ist, besteht die Gefahr, dass ungewünschte Lücken oder Textabschnitte entstehen. In verschiedenen Sprachen ist auch die Satzstellung verschieden. Hat man nun einen Satz mit einer eingesetzten Variable aus dem Programm, ändert sich die Satzstellung. Ein Beispiel für Deutsch gegenüber Englisch wäre: "vor {Variable} Tagen" soll übersetzt "{Variable} days ago" heißen. Ein Lösungsansatz für dieses Problem ist die Verwendung von Platzhaltern.\autocite{wikii18n}