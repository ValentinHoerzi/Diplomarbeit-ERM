\chapter{Zusammenfassung und Ausblick}
\section{Zusammenfassung}
Im Rahmen dieser Diplomarbeit wurde das Endkundernportal erweitert. Es wurde eine i18n und dynamische Registrierungsmöglichkeit entwickelt, mit der man Zugang zu einem eigenen Maschinenpark erhält. Dieses Registrierungsformular richtet sich nach dem ausgewählten Land und der dazugehörigen Sprache und besitzt ebenfalls eine vollkommen dynamische Validierung. Durch den Maschinenpark hat der Benutzer einen Schnellzugriff auf all seine gespeicherten Maschinen, wo er alle Informationen über diese entnehmen kann. Für einen reibungslosen Ablauf von der Programmierung bis hin zur Veröffentlichung der Software wurde eine komplette CI/CD Lösung entwickelt.

\section{Ausblick}
Dieses Endkundenportal ist in Zukunft noch um einige Features erweiterbar. Um so mehr sich die eingebaute Technik in den landwirtschaftlichen Maschinen erhöht, desto mehr Funktionen sind auch hier implementierbar. Eine Möglichkeit ist die Einsicht in Live-Daten der Maschinen. Somit soll es möglich sein, die Betriebszeit, die gefahrenen Kilometer oder den Standort der eigenen Maschine herauszulesen. Weiters ist im derzeitigen Stand des Projektes der FAQ-Bereich nur im Backend vorhanden. Das bedeutet, dass ein Frontend dafür erstellt werden kann.