\documentclass[12pt,a4paper,oneside,ngerman,
%draft
]{report}

%Packages
\usepackage[utf8]{inputenc}
\usepackage[T1]{fontenc}

\usepackage[table, dvipsnames]{xcolor}
\usepackage{multicol}

\usepackage{amsmath}
\usepackage{amsfonts}
\usepackage{amssymb}
\usepackage{graphicx}
\usepackage[export]{adjustbox} %this is for frames on images
\usepackage{fontspec}
\usepackage{titlesec}
\usepackage{tocloft}
\usepackage[margin=2.5cm]{geometry}
\usepackage{lipsum}
\usepackage[skip=10pt]{parskip}
\usepackage{mwe,tikz}
\usepackage{tabto}
\usepackage{dashrule}
\usepackage[backend=biber,style=numeric]{biblatex}
\usepackage{listings,lstautogobble}
\usepackage{scrextend}
\usepackage{enumitem}
\usepackage[hidelinks]{hyperref} %hidelinks removes red borders around hyperlinks
\usepackage{float}
\usepackage{wrapfig}
\usepackage{makecell}
\usepackage[german,english]{babel}
\usepackage{pdfpages}
\usepackage{tabularx}

\usepackage[onehalfspacing]{setspace}

\usepackage{fancyhdr}
\pagestyle{fancy}
\fancyhead{}
\fancyfoot{}
\renewcommand{\headrulewidth}{0pt}
\renewcommand{\footrulewidth}{0.4pt}
\rfoot{\thepage}
\lfoot{Endkundenportal: Registrierung und Maschinenpark}
\lhead{}
\rhead{}



%Commands
\newcommand{\ThRealAuthorNameOne}{Valentin Hörzi}
\newcommand{\ThRealAuthorNameTwo}{Alexander Schörgendorfer}
\newcommand{\ThRealAuthorOneEmail}{valentin.hoerzi@gmail.com}
\newcommand{\ThRealAuthorTwoEmail}{alex.schoergi@gmail.com}
\newcommand{\ThAuthorsClass}{5AHIF}
\newcommand{\ThAuthorOneNumber}{13}
\newcommand{\ThAuthorTwoNumber}{22}
\newcommand{\ThPartner}{Pöttinger Landtechnik GmbH, Industriestraße 1, 4710 Grieskirchen}
\newcommand{\ThPartnerName}{Pöttinger Landtechnik GmbH}
\newcommand{\ThPartnerPersonName}{Dominik Augustin}
\newcommand{\ThPartnerPersonEmail}{Dominik.Augustin@poettinger.at}
\newcommand{\ThSchoolName}{HTBLA Grieskirchen}
\newcommand{\ThPhysicalLocation}{Grieskirchen}
\newcommand{\ThSupervisorName}{Mag. Dipl.-Ing. Rainer Sickinger}
\newcommand{\ThSupervisorEmail}{sickinger@docsced.at}

\renewcaptionname{english}{\figurename}{Abb.}
\renewcaptionname{english}{\tablename}{Tab.}


\addbibresource{citations.bib}

\setcounter{secnumdepth}{3}
\addtolength{\textwidth}{-1cm}
\addtolength{\oddsidemargin}{1cm}
\addtolength{\evensidemargin}{1cm}

\lstset{autogobble=true}
\lstset{language=sh,basicstyle=\ttfamily}
\lstset{language=,basicstyle=\ttfamily}

\definecolor{MyLightGrayBackgroundForCode}{RGB}{227,227,227}

% C# listing
\definecolor{bluekeywords}{rgb}{0,0,1}
\definecolor{greencomments}{rgb}{0,0.5,0}
\definecolor{redstrings}{rgb}{0.64,0.08,0.08}
\definecolor{xmlcomments}{rgb}{0.5,0.5,0.5}
\definecolor{types}{rgb}{0.17,0.57,0.68}
\lstset{language=[Sharp]C,	
	%backgroundcolor=\color{black!10}
	captionpos=b,
	numbers=left, %Nummerierung
	%numberstyle=\tiny, % kleine Zeilennummern
	%frame=lines, % Oberhalb und unterhalb des Listings ist eine Linie
	showspaces=false,
	showtabs=false,
	breaklines=true,
	showstringspaces=false,
	breakatwhitespace=true,
	escapeinside={(*@}{@*)},
	morekeywords={partial, var, value, get, set},
	keywordstyle=\color{bluekeywords},
	commentstyle=\color{greencomments},
	stringstyle=\color{greencomments},
	basicstyle=\ttfamily\small,
	backgroundcolor=\color{MyLightGrayBackgroundForCode},
	tabsize=4,
}
% End of C# listing

%TypeScript listing	, class, private, constructor, string, T
\lstdefinelanguage{JavaScript}{
	morekeywords=[1]{break, continue, delete, else, for, function, class, constructor, if, in,
		new, return, this, typeof, var, void, while, with,const},
	% Literals, primitive types, and reference types.
	morekeywords=[2]{false, null, true, boolean, number, undefined, private, public,export, declare, interface,any,
		Array, Boolean, Date, Math, Number, String, string, T, Object},
	% Built-ins.
	morekeywords=[3]{eval, parseInt, parseFloat, escape, unescape},
	sensitive,
	morecomment=[s]{/*}{*/},
	morecomment=[l]//,
	morecomment=[s]{/**}{*/}, % JavaDoc style comments
	morestring=[b]',
	morestring=[b]",
	backgroundcolor=\color{MyLightGrayBackgroundForCode},
	tabsize=4,
}[keywords, comments, strings]
%End of TypeScript listing

%Json listing
\definecolor{delim}{RGB}{20,105,176}
\definecolor{numb}{RGB}{106, 109, 32}
\definecolor{string}{rgb}{0.64,0.08,0.08}

\lstdefinelanguage{json}{
	captionpos=b,
	numbers=left,
	%numberstyle=\small,
	%frame=single,
	%rulecolor=\color{black},
	showspaces=false,
	showtabs=false,
	breaklines=true,
	showstringspaces=false,
	postbreak=\raisebox{0ex}[0ex][0ex]{\ensuremath{\color{gray}\hookrightarrow\space}},
	breakatwhitespace=true,
	basicstyle=\ttfamily\small,
	upquote=true,
	morestring=[b]",
	stringstyle=\color{string},
	backgroundcolor=\color{MyLightGrayBackgroundForCode},
	tabsize=4,
	literate=
	*{0}{{{\color{numb}0}}}{1}
	{1}{{{\color{numb}1}}}{1}
	{2}{{{\color{numb}2}}}{1}
	{3}{{{\color{numb}3}}}{1}
	{4}{{{\color{numb}4}}}{1}
	{5}{{{\color{numb}5}}}{1}
	{6}{{{\color{numb}6}}}{1}
	{7}{{{\color{numb}7}}}{1}
	{8}{{{\color{numb}8}}}{1}
	{9}{{{\color{numb}9}}}{1}
	{\{}{{{\color{delim}{\{}}}}{1}
	{\}}{{{\color{delim}{\}}}}}{1}
	{[}{{{\color{delim}{[}}}}{1}
	{]}{{{\color{delim}{]}}}}{1}
}
%End of Json listing

%YAML listing
\lstdefinelanguage{yaml}{
	keywords={image, stage, script, tags, artifacts, paths, expire_in, rules,name,cache,services,before_script},
	keywordstyle=\color{blue}\bfseries,
	identifierstyle=\color{black},
	sensitive=false,
	comment=[l]{\#},
	commentstyle=\color{purple}\ttfamily,
	stringstyle=\color{red}\ttfamily,
	morestring=[b]',
	morestring=[b]",
	backgroundcolor=\color{MyLightGrayBackgroundForCode},
	tabsize=4,
}

\lstset{basicstyle=\ttfamily,
	showstringspaces=false,
	commentstyle=\color{red},
	keywordstyle=\color{blue},
	inputencoding=utf8,
	extendedchars=true
}

%%End of YAML listing

% Calibri or sans-serif as default font
\defaultfontfeatures{Mapping=tex-text,Scale=MatchLowercase}
\IfFontExistsTF{Calibre}{
	\setmainfont{Calibre}
}{
	\renewcommand{\familydefault}{\sfdefault}
}

% TOC horizontal spacing
\cftsetindents{chapter}{0cm}{1cm}
\cftsetindents{section}{1cm}{1cm}
\cftsetindents{subsection}{2cm}{1.2cm}

% TOC vertical spacing
\renewcommand{\cftchapfont}{
	\bfseries
	\fontsize{13pt}{13pt}
	\selectfont
	\vspace{6pt}
}
\cftbeforechapskip12pt
\renewcommand{\cftsecfont}{
	\fontsize{11pt}{11pt}
	\selectfont
	\vspace{6pt}
}
\cftbeforesecskip6pt
\renewcommand{\cftsubsecfont}{
	\fontsize{10pt}{10pt}
	\selectfont
	\vspace{3pt}
}
\cftbeforesubsecskip6pt

% TOC title
\renewcommand\cfttoctitlefont{
	\hfill
	\fontsize{16pt}{16pt}
	\selectfont
	\bfseries
}
\cftbeforetoctitleskip6pt
\cftaftertoctitleskip12pt

% Title format for chapter, section, sub- and subsubsection
% See https://tex.stackexchange.com/questions/511981/titlesec-vertical-align-chapter-and-section
% The -5pt is just pixel pushing bc I couldn't figure out why my chapter/section/subsection/subsubsection headings are slightly indented
\titleformat{\chapter}[hang]{
	\normalfont
	\bfseries
	\filright
	\fontsize{16pt}{16pt}
	\selectfont
	\thispagestyle{fancy}
}{
	\makebox[1.7cm][l]{\thechapter}
}{0em}{}
\titlespacing{\chapter}{-5pt}{18pt}{12pt}

\titleformat{\section}[hang]{
	\normalfont
	\bfseries
	\filright
	\fontsize{14pt}{14pt}
	\selectfont
}{
	\makebox[1.7cm][l]{\thesection}
}{0em}{}
\titlespacing{\section}{-5pt}{18pt}{12pt}

\titleformat{\subsection}[hang]{
	\normalfont
	\bfseries
	\filright
	\fontsize{12pt}{12pt}
	\selectfont
}{
	\makebox[1.7cm][l]{\thesubsection}
}{0em}{}
\titlespacing{\subsection}{-5pt}{18pt}{12pt}

\titleformat{\subsubsection}[hang]{
	\normalfont
	\bfseries
	\filright
	\fontsize{11pt}{11pt}
	\selectfont
}{
	\makebox[1.7cm][l]{\thesubsubsection}
}{0em}{}
\titlespacing{\subsubsection}{-5pt}{12pt}{12pt}

% Have list of figures title style be the same as TOC title stile
\renewcommand{\cftloftitlefont}{\cfttoctitlefont}

% list of tables title style same as toc title style
\renewcommand{\cftlottitlefont}{\cfttoctitlefont}

\newcommand{\SignatureLine}[1]{
	\vskip15pt
	\tabto{9cm}#1
	\vskip10pt
	\tabto{9cm}\hdashrule[0pt][x]{\fill}{.5pt}{.75mm}
}

\newcounter{appendix}
\newcommand{\Appendix}[1]{
	\stepcounter{appendix}
	\chapter*{\hspace{\fill}Anhang \Alph{appendix}} \label{app:\Alph{appendix}}
	#1
}

\newcommand{\RefAppendix}[1]{\hyperref[app:#1]{Appendix~#1}}


\title{ERM-Diplomarbeit}
\author{Valentin Hörzi, Alexander Schörgendorfer}

	%%%%%%%%%%%%%%%%%%%%%%%%%%%%%%%%%%%%%%%%%
	%%%   Beginn   %%%%%%%%%%%%%%%%%%%%%%%%%%
	%%%%%%%%%%%%%%%%%%%%%%%%%%%%%%%%%%%%%%%%%

\begin{document}
	
	\pagestyle{empty}
	
	%%%%%%%%%%%%%%%%%%%%%%%%%%%%%%%%%%%%%%%%%
	%%%   TITELBLATT   %%%%%%%%%%%%%%%%%%%%%%
	%%%%%%%%%%%%%%%%%%%%%%%%%%%%%%%%%%%%%%%%%
	\newcommand{\IncludeSchoolTemplate}[2]{
	\vspace*{-7em}
	\makebox[\textwidth]{
		\begin{tikzpicture}[
			every node/.style={anchor=north west,inner sep=0pt},
			x=1mm, y=1mm]
			\node (templatepage) at (0,0)
			{\includegraphics[width=\paperwidth,page=#1]{./summary.pdf}};
			#2
		\end{tikzpicture}
	}
	\newpage
}

\pagenumbering{gobble}

\selectlanguage{german}
\includegraphics[width=\textwidth]{./grafiken/school-header.png}
{\centering
	\vskip1cm
	Fachrichtung Informatik
	\vskip2cm
	Schuljahr 2020/21
	\vskip4cm
	\Huge\textbf{ERM}
	\vskip10pt
	\large
	Gesamtprojekt
	\vskip5pt
	\Huge\textbf{Endkundenportal: Registrierung und Maschinenpark}
	\small
	\vskip4cm
	\begin{flushleft}
		\textbf{Ausgeführt von:}\tabto{9cm}\textbf{Betreuer/Beteuerin:}\linebreak
		\ThRealAuthorNameOne, \ThAuthorsClass-\ThAuthorOneNumber\tabto{9cm}\ThSupervisorName\linebreak
		\ThRealAuthorNameTwo, \ThAuthorsClass-\ThAuthorTwoNumber
		\vskip1cm
		\ThPhysicalLocation, am \today
		\vskip1cm
		\hrule
		Abgabevermerk:\linebreak
		Datum:\tabto{10cm}Betreuer/Beteuerin:
	\end{flushleft}
}



\selectlanguage{german}
\chapter*{\hspace{5pt}Erklärung gemäß Prüfungsordnung}
„Ich erkläre an Eides statt, dass ich die vorliegende Diplomarbeit selbstständig und ohne fremde Hilfe verfasst, andere als die angegebenen Quellen und Hilfsmittel nicht benutzt und alle den benutzten Quellen wörtlich oder sinngemäß entnommenen Stellen als solche kenntlich gemacht habe.“

\ThPhysicalLocation, \today\tabto{9cm}{Verfasser*innen:}
\selectlanguage{english}

\SignatureLine{\ThRealAuthorNameOne}
\SignatureLine{\ThRealAuthorNameTwo}
\newpage
\IncludeSchoolTemplate{1}{
	\node at (77,-28)
		{Informatik};
	\node at (77,-63)
		{\ThRealAuthorNameOne};
	\node at (77,-70)
		{\ThRealAuthorNameTwo};
	\node at (77,-80)
		{2020/21};
	\node at (77,-92)
		{Beispielarbeit};
	\node at (77,-109)
		{\ThPartner};
	\node at (77,-127) [text width=305,align=justify] {
		\fontsize{12pt}{12pt}
		\selectfont
		\par
		Ziel des Projekts war die Entwicklung eines Tools zu \lipsum[1][1-3]
		\vskip10pt
		\par
		Alle Programme sollten in Java realisiert werden.
	};
	\node at (77,-174) [text width=300,align=justify] {
		\fontsize{12pt}{12pt}
		\selectfont
		\par
		\lipsum[2][1-3]
		\vskip10pt
		\par
		\lipsum[3][1-3]
	};
	\node at (77, -222) [text width=300,align=justify] {
		\fontsize{12pt}{12pt}
		\selectfont
		\par
 \lipsum[4]
	};
}
\IncludeSchoolTemplate{2}{
	\node at (77,-28)
		{Informatik};
	\node at (77,-80){
		\includegraphics[width=310pt,trim=0pt 0pt 0pt 0pt,clip]{./grafiken/sample_graphic.png}
	};
	\node at (77,-171) [text width=305,align=center]
		{\emph{Screenshot des Programms}};
	\node at (77,-220)
		{--};
	\node at (77,-239)
		{Öffentlich; Bibliothek der \ThSchoolName};
}
\IncludeSchoolTemplate{3}{
	\node at (76,-28)
		{Informatics};
	\node at (77,-63)
		{\ThRealAuthorNameOne};
	\node at (77,-80)
		{2019/20};
	\node at (77,-92)
		{Sample Project};
	\node at (77,-109)
		{\ThPartner};
	\node at (77,-125) [text width=305,align=justify] {
		\fontsize{12pt}{12pt}
		\selectfont
		\par
		\lipsum[1][1-5]
	};
	\node at (77,-173) [text width=300,align=justify] {
		\fontsize{12pt}{12pt}
		\selectfont
		\par
		\lipsum[5][1-5]
	};
	\node at (77, -225) [text width=300,align=justify] {
		\fontsize{12pt}{12pt}
		\selectfont
		\par
		\lipsum[6][1-8]
	};
}
\IncludeSchoolTemplate{4}{
	\node at (76,-28)
		{Informatics};
	\node at (77,-75){
		\includegraphics[width=310pt,trim=0pt 0pt 0pt 0pt,clip]{./grafiken/sample_graphic.png}
	};
	\node at (77,-166) [text width=305,align=center]
		{\emph{Screenshot of the program}};
	\node at (77,-220)
		{--};
	\node at (77,-240)
		{Public; Library of the \ThSchoolName};
}
	
	\clearpage
	\renewcommand{\contentsname}{Inhaltsverzeichnis}
	\tableofcontents
	\newpage
	\pagestyle{fancy}
	\pagenumbering{arabic}
	
	%%%%%%%%%%%%%%%%%%%%%%%%%%%%%%%%%%%%%%%%%
	%%%   EINLEITUNG   %%%%%%%%%%%%%%%%%%%%%%
	%%%%%%%%%%%%%%%%%%%%%%%%%%%%%%%%%%%%%%%%%
	
	\chapter{Introduction} \label{sec:introduction}

% Example of citing some website
This thesis aims to do the thing\parencite{somewebsite}.

% Example of featuring a graphic
\begin{figure}[h]
	\includegraphics[width=\textwidth]{./grafiken/sample_graphic.png}
	\vskip0pt
	\caption{Screenshot of the program} \label{fig:samplefigure}
\end{figure}

% Example of featuring codesnippets
\begin{lstlisting}[language=Java]
	public static void main() {
		System.out.println("Hello World");
	}
\end{lstlisting}

See \autoref{fig:samplefigure} and \autoref{sec:introduction}~Introduction.



	
	%%%%%%%%%%%%%%%%%%%%%%%%%%%%%%%%%%%%%%%%%
	%%%   Projektmanagement   %%%%%%%%%%%%%%%%%%%%%%%%
	%%%%%%%%%%%%%%%%%%%%%%%%%%%%%%%%%%%%%%%%%
	
	\chapter{Projektmanagement}
Um die Diplomarbeit ohne große Komplikationen durchführen zu können, ist das Projektmanagement ein wichtiger Bestandteil davon. Im November 2019 haben wir begonnen uns mit der Firma Pöttinger Landtechnik GmbH und unserem Betreuer Herrn Rainer Sickinger abzusprechen.\\
Anschließend erfolgte die Erstellung des Pflichtenheftes, welches den Start der Projektarbeit darstellte. Finalisiert wurde das Pflichtenheft in Zusammenarbeit mit \ThPartnerPersonName  in einem Meeting. 
%\rowcolor{lightgray}
\section{Meilensteine}
{\rowcolors{2}{gray!20}{gray!10}
\begin{tabular}[h]{|l|p{8cm}|p{2.5cm}|p{2.5cm}|}
	\hline
	\# & Meilenstein & SOLL-Fertigstellung & IST-Fertigstellung \\
	\hline
	1 & Erstellung eines statischen Registrierungsformulars & Juli 2020 & Juli 2020 \\
	\hline
	2 & Aufbau einer dynamischen Internationalisierungsmöglichkeit & Juli 2020 & Juli 2020 \\	
	\hline
	3 & Erweiterung durch Komfortfeatures zur schnelleren/benutzerfreundlicheren Registrierung & Juli 2020 & Juli 2020 \\
	\hline
	4 & Schaffen von Login/Logout-Möglichkeiten & Juli 2020 & Juli 2020 \\	
	\hline
	5 & Entwicklung einer dynamisch generierten Registrierungsformulars & Juli 2020 & Juli 2020 \\	
	\hline
	6 & Gestaltung und Implementierung einer Profilseite zur Einsicht und Bearbeitung der Daten, sowie An-/Abmeldung zum Newsletter & Juli 2020 & Juli 2020 \\	
	\hline
	7 & Entwicklung des Maschinenparks & Juli 2020 & Juli 2020 \\	
	\hline
	8 & Einbinung der bereits existierenden Module in die Startseite (Produktpalette / Maschinensuche / Konfigurator / PÖTSEM) & August 2020 & August 2020 \\	
	\hline
	9 & Implementierung der Startseite & August 2020 & August 2020 \\	
	\hline
	10 & CI/CD Pipeline & Jänner 2021 & Februar 2021 \\	
	\hline	
\end{tabular}
	
	%%%%%%%%%%%%%%%%%%%%%%%%%%%%%%%%%%%%%%%%%
	%%%   VERWENDETE TECHNOLOGIEN   %%%%%%%%%
	%%%%%%%%%%%%%%%%%%%%%%%%%%%%%%%%%%%%%%%%%
	
	\chapter{Verwendete Technologien} \label{technologien}
\section{Twilio} \label{sec:twilio}
Twilio wurde 2008 von Jeff Lawson, Evan Cooke und John Wolthuis in Amerika gegründet. Es betreibt eine Cloud-Kommunikationsplattform als Platform as a Service. Mit den von Twilio zur Verfügung gestellten Dienst, können Entwickler Programmierschnittstellen zum ausführen und empfangen von Anrufen, senden von SMS, verifizieren von Telefonnummern sowie für andere Kommunikationsfunktionen nutzen. In unseren Fall wurde mit Twilio überprüft, ob der Nutzer eine gültige Telefonnummer eingegeben hat. \cite{twilioWebsite}

\section{Address Validator}
Der Address Validator wurde von Byteplant Software Solutions \& Services entwickelt. Dieses Unternehmen wurde 2003 in Deutschland gegründet. Neben dem Address Validator bieten sie auch einen Email Validator und einen Phone Validator an. Durch die Addressvalidierung erhält Informationen über die Zustellbarkeit, Korrekturvorschläge falls die Adresse nicht ganz gültig ist und eine einheitliche Adressformatierung nach den nationalen Standards. Für eine Überprüfung benötigt man Straße, Stadt, Internationales Länderkürzel und den API-Key. Optional kann man noch mehr, wie zum Beispiel Postleitzahl oder die Locale in der man das Ergebnis erhalten möchte. 
Als Antwort bekommt man ein JSON. Aus diesem JSON bekommt man viele Informationen über die Adresse. Falls die Adresse gültig ist, bekommt man im Feld Status ein VALID, bei ungültiger Adresse ein INVALID und falls die Überprüfung kein genaues Ergebnis herausgefunden hat, aber eine Adresse gefunden hat, die es sein könnte, ist der Status SUSPECT mit einen Adressvorschlag. Weiters beinhaltet die Antwort alles von den einzelnen Adressfelder bis hin zu der formatierten Adresse des Landes und den Geografischen Koordinaten. \cite{addressValidator}

\section{Google Translator}
Die Schnittstelle für den Google Cloud Translator wurde von Google LLC entwickelt. Damit kann man schnell Texte in mehr als 100 Sprachen im eigenen Programm übersetzen lassen. Als Antwort der Übersetzung bekommt man den Titel, den Content und die Abkürzung der Sprache. Nun kann man die Übersetzungen speichern und man braucht nur nach der Spachenabkürzung suchen.
\cite{googleTranslator}
	
	%%%%%%%%%%%%%%%%%%%%%%%%%%%%%%%%%%%%%%%%%
	%%%   INTERNATIONALISIERUNG   %%%%%%%%%%%
	%%%%%%%%%%%%%%%%%%%%%%%%%%%%%%%%%%%%%%%%%
	
	\chapter{Internationalisierung}
In der Informatik bezeichnet man ein Programm als internationalisiert, wenn es mit dem unveränderten Quellcode in andere Sprachen und Kulturen übersetzt werden kann. Diese Bezeichnung wird oftmals als i18n abgekürzt. Die Zahl 18 steht für die Anzahl der Buchstaben, die sich zwischen dem ersten und den letzten Buchstaben im englischen Wort "internationalization" stehen. \autocite{wikii18n}
\section{Umsetzung}
Damit für eine Übersetzung der Quellcode nicht verändert werden muss, darf ein Text nicht hartkodiert sein. Für die verwendeten Texte sollen 
	
	%%%%%%%%%%%%%%%%%%%%%%%%%%%%%%%%%%%%%%%%%
	%%%   CQS   %%%%%%%%%%%%%%%%%%%%%%%%%%%%%
	%%%%%%%%%%%%%%%%%%%%%%%%%%%%%%%%%%%%%%%%%
	
	\chapter{Command-query Separation}
Das Prinzip der Command-query Separation (CQS) steht für die Trennung von Befehlen und Abfragen. In der klassischen Geschäftlogik-Schicht gibt es mehrere Schnittstellen, die verschiedenste Aufgaben abdecken. Da die Möglichkeit besteht, dass  mit der Zeit immer mehr Anforderungen an das System gestellt werden, werden die Schnittstellen und die darin befindenden Methoden immer mehr und möglicherweise auch die Cross-Cutting Concerns. Mit Cross-Cutting Concerns sind Anforderungen gemeint, die das ganze System betreffen, aber nicht von Anwendungsfall. Das können Transaktionen, Validierungen oder Rechteüberprüfungen sein. In dieser Diplomarbeit wurde bei jeden Aufruf geprüft, ob der Benutzer die berechtigten Rechte hat. Somit wäre ohne CQS in jeder Methode etwas zu ändern. \autocite{cqsSOLIDeArchitektur}\\
Um das umzusetzen, werden die zwei generische Schnittstellen benötigt, die man in Listing \ref{lst:cqsSchnittstellen} sieht. \texttt{IQuery} steht für Abfragen und \texttt{ICommand}, die für Aktionen steht.
\begin{lstlisting}[caption={CQS-Schnittstellen},captionpos=b, numbers=left, backgroundcolor=\color{black!10},language={[Sharp]C}, label={lst:cqsSchnittstellen}]
	public interface IQuery<TResult> { }
	public interface ICommand { }
\end{lstlisting}
Eine Methode aus dem Controller wird nun eine eigene Klasse. Als Eigenschaften dieser neuen Klasse werden die Methodenparameter verwendet. Ein Beispiel aus dieser Diplomarbeit kann man im Listing \ref{lst:getFaqbyid} sehen. Es implementiert dass im Listing \ref{lst:cqsSchnittstellen} vorhandene Interface \texttt{IQuery} mit dem Datentyp FAQDto.
\begin{lstlisting}[caption={CQS-Query Beispiel},captionpos=b, numbers=left, backgroundcolor=\color{black!10},language={[Sharp]C}, label={lst:getFaqbyid}]
	public class GetFAQById : IQuery<FAQDto>
	{
		public GetFAQById(Guid faqId)
		{
			this.FAQId = faqId;
		}
		public Guid FAQId { get;}
	}
\end{lstlisting}
Um die gewollte Logik auszuführen, werden zwei weitere generischen Schnittstellen benötigt. Wiederum gitb es eine für Abfragen und eine für Aktionen wie man im Listing \ref{lst:cqsHandler} sehen kann. Diese werden als Handler betitelt. \autocite{cqsSOLIDeArchitektur}
\begin{lstlisting}[caption={CQS-Query Beispiel},captionpos=b, numbers=left, backgroundcolor=\color{black!10},language={[Sharp]C}, label={lst:cqsHandler}]
	public interface IQueryHandler<TQuery, TResult>	where TQuery : IQuery<TResult>
	{
		Task<TResult> HandleAsync(TQuery query);
	}
	
	public interface ICommandHandler<TCommand> where TCommand : ICommand
	{
		Task HandleAsync(TCommand command);
	}
\end{lstlisting}
	
	%%%%%%%%%%%%%%%%%%%%%%%%%%%%%%%%%%%%%%%%%
	%%%   FAQ-Backend   %%%%%%%%%%%%%%%%%%%%%
	%%%%%%%%%%%%%%%%%%%%%%%%%%%%%%%%%%%%%%%%%

	\chapter{FAQ}
\section{Was ist FAQ}
FAQ steht für "Frequently Asked Questions". Es handelt sich dabei um eine Ansammlung von häufig gestellten Fragen, die auch für weitere Benutzer interessant sein könnten. Somit findet man auf einer Webseite im Bereich FAQ eine Auflistung dieser Ansammlung der wiederkehrenden Benutzerfragen.
\section{FAQ-Backend}
Eine der Aufgaben war, ein funktionierendes Backend für einen zukünftigen FAQ-Bereich zu programmieren. Wenn eine Person, eine Frage gestellt bekommt und diese bearbeitet, beantwortet und dabei empfindet, dass diese Frage auch noch für weitere Benutzer von Interesse sein könnte und diese Person auch die erforderlichen Rechte dafür besitzt, kann sie aus dieser Frage ein FAQ erstellen.\\
Dafür wurde zuerst eine Tabelle in der Datenbank definiert. Ein FAQ besteht in der Datenbank aus einer ID, einem Sprachenkürzel, dem Titel, den Inhalt der Frage, einem Zähler, der die Abfragen mitzählt und einer Liste mit den vorhandenen Übersetzungen für diese Frage. Die Berechtigung ist mit einem Claim in dem JWT gespeichert. Wenn nun der Befehl um eine FAQ zu erstellen aufgerufen wird, wird zuerst überprüft, ob die benötigte Berechtigung vorhanden ist. Falls vorhanden, wird die FAQ erstellt und automatisch mit Google Cloud Translate übersetzt und gespeichert. In diesen folgenden Sprachen ist die FAQ danach übersetzt:
\begin{multicols}{2}
	\begin{itemize}
		\item Deutsch
		\item Englisch
		\item Französisch
		\item Niederländisch
		\item Dänisch
		\item Spanisch
		\item Italienisch
		\item Ungarisch
		\item Polnisch
		\item Slowakisch
		\item Finnisch
		\item Schwedisch
		\item Tschechisch
		\item Russisch
		\item Ukrainisch
	\end{itemize}
\end{multicols}
	
	%%%%%%%%%%%%%%%%%%%%%%%%%%%%%%%%%%%%%%%%%
	%%%   REGISTRIERUNGSFORMULAR   %%%%%%%%%%
	%%%%%%%%%%%%%%%%%%%%%%%%%%%%%%%%%%%%%%%%%
	
	\chapter{Das Registrierungsformular}

Der wohl komplizierteste Teil dieser Diplomarbeit ist der Aufbau und die korrekte Validierung des Registrierungsformulars. Das Ziel ist den Benutzern eine möglichst einfache und schnelle Registrierung, unter Berücksichtigung des derzeitigen Standorts und der dazugehörigen Benutzergruppe, anzubieten. Dafür wurde ein Algorithmus entwickelt, welcher bei der \texttt{ngAfterViewInit()}-Lifecycle-Hook-Methode im \texttt{UserprofileComponent} seinen Start nimmt.

\section{Der Algorithmus hinter dem dynamischen Aufbau}

\begin{figure}[H]
	\centerline{
		\includegraphics[width=1\textwidth, frame]{./grafiken/RF_Flussdiagramm.png}
	}
	\vskip0pt
	\caption{Flussdiagramm des Algorithmus}
	\label{fig:fc}
\end{figure}

\subsection{\texttt{initiateDynamicComponentConfiguration()}} \label{ssec:lblInitDCC}

Wenn das User-Objekt, welches von der Backend-API abgefragt und erfolgreich als globale Objektvariable gespeichert wurde, wird die \texttt{initiateDynamicComponent-\\Configuration()}-Methode aufgerufen. Aus Performance-Gründen basiert die nächste Überprüfung auf dem Wert der aktuellen Lokale. Gleicht dieser den Wert "AT", so wird ein vorgefertigtes Template für ein österreichisches RF, dargestellt im Listing~\ref{lst:template_form_aut}, für die Weiterverwendung benützt. Da davon ausgegangen wird, dass sich vor allem in der Anfangsphase nach der Veröffentlichung hauptsächlich österreichische Kunden ein Konto erstellen, spart dieser Weg enorm viel Zeit, da sofort mit der Erstellung des RF begonnen werden kann.

\begin{lstlisting}[caption={Vordefiniertes Template für das RF}, language=JavaScript,label={lst:template_form_aut}]
export const AUSTRIAN_PRIVATE_PERSON_FORM_Template = [
	[{ type: GenderComponent }],
	[
		{ type: FirstnameComponent, options: { label: "Vorname" } },
		{ type: LastnameComponent, options: { label: "Nachname" } },
	],
	[{ type: StreetComponent, options: { label: "Straße und Hausnummer" } }],
	[
		{ type: CityComponent, options: { label: "Stadt" } },
		{ type: PostalcodeComponent, options: { label: "Postleitzahl" } },
	],
	[
	{
		type: PhoneComponent,
		options: { label: "Telefonnummer", phonePrefix: 43 },
	},
	{ type: DateofbirthComponent },
	],
];
\end{lstlisting}

\subsection{\texttt{handleTemplateCall()}}

Anderenfalls wird als nächster Schritt die \texttt{handleTemplateCall()}-Methode, markiert im Flussdiagramm~\ref{fig:fc} mit [1], aufgerufen, welche, basierend auf dem Wert der aktuellen Lokale, auf die Shopify-API zugreift, um ein \texttt{Country}-Objekt abzurufen. Wurde diese Operation beendet, folgt ein Update auf eine Map, welche vorläufig die Labels für die Wrapper-Objekte speichert. Nachdem das Template geladen wurde, wird über jedes einzelne Element iteriert, markiert im Flussdiagramm~\ref{fig:fc} mit [2], um daraus eine oder mehrere Zeilen zu generieren. Anschließend werden diese in ein 2D-Array gespeichert, welches dann der \texttt{generateDynamicView()}-Methode zur Generierung des RF übergeben wird.

\subsection{\texttt{analyzeRow()}} \label{ssec:lblAnalyzeRow}

\begin{lstlisting}[caption={Erstellung des 2D-Arrays für den Aufbau des RF}, language=JavaScript,label={lst:analyzeRow}]
private analyzeRow(row: string[]): RegisterComponent[][] {
	let addGenderC = false;
	let result: RegisterComponent[][] = [];
	let componentRow: RegisterComponent[] = [];
	
	for (let i = 0; i < row.length; i++) {
		let rowEl = row[i];
		
		if (!addGenderC) {
			if (rowEl.includes("firstName") || rowEl.includes("lastName")) {
				let genderComponent: RegisterComponent[] = [
				{ type: GenderComponent },
				];
				result.push(genderComponent);
				addGenderC = true;
			}
		}
		
		if (rowEl.includes("company") && this.user.isPrivatePerson) {
			break;
		}
		
		for (const x of this.valueMap.keys()) {
			if (rowEl.includes(x)) {
				let compEl: RegisterComponent = {
					type: this.valueMap.get(x),
				};
				
				if (
				compEl.type.prototype ===
				ZoneComponent.prototype
				) {
					compEl.options.zones = this.countryFromService.zones.map(
					(x) => x.name
					);
					this.user.zone = compEl.options.zones[0];
				}
				
				if (
				compEl.type.prototype ===
				PhoneComponent.prototype
				) {
					compEl.options.phonePrefix = this.countryFromService.phoneNumberPrefix;
				}
				
				if (this.labels.get(x)) {
					compEl.options.label = this.labels.get(x);
				}
				
				componentRow.push(compEl);
				if (rowEl.includes("phone")) {
					let dateofbirthComponent: RegisterComponent = {
						type: DateofbirthComponent,
					};
					componentRow.push(dateofbirthComponent);
				}
				break;
			}
		}
	}
	result.push(componentRow);
	return result;
}
\end{lstlisting}

Die im Listing~\ref{lst:analyzeRow} beschriebene \texttt{analyzeRow()}-Methode nimmt ein Array von Zeichenfolgen als Parameter und gibt ein 2D-Array von \texttt{RegisterComponent}-Wrapper-Objekten zurück. Obwohl nur eine Reihe aus dem Template, also z. B. \texttt{[\{firstName\} \{lastname\}]}, überprüft wird, kann es wie in diesem Fall sein, dass ein Input zusätzlich darüber erzeugt werden muss. So wird in Zeile 9 überprüft, ob der \texttt{GenderComponent} schon hinzugefügt wurde. Trifft das und die Überprüfung, ob in diesem Durchgang der \texttt{FirstnameComponent} oder der \texttt{LastnameComponent} erzeugt werden, wird der \texttt{GenderComponent} als erster eingefügt. Dieser ist für die Auswahl der Anrede zuständig. 

Der selbe Mechanismus kann auch in Zeile 51 gefunden werden, wo es nötig ist, den \texttt{DateofbirthComponent}, welcher für die Eingabe des Geburtstages verantwortlich ist, neben dem \texttt{PhoneComponent} zu generieren.

Anschließend wird in Zeile 19 sicher gestellt, dass eine Privatperson keine Auswahl einer Firma präsentiert wird.

Um nun tatsächlich aus dem Template Objekte zu erzeugen, wird ab Zeile 23 über Schlüssel einer Map, beschrieben im Listing~\ref{lst:valueMap}, welche den String eines Templates als \texttt{key} und den korrespondierenden Component-Objekten als \texttt{value} hat, iteriert. 

\begin{lstlisting}[caption={valueMap in der \texttt{analyzeRow()}-Methode}, language=JavaScript,label={lst:valueMap}]
export class ValueMapper {
	static COMPONENT_VALUE_MAP: Map<string, Type<any>> = new Map();
	
	static COMPONENT_VALUE_MAPPER(){
		this.COMPONENT_VALUE_MAP.set('firstname', FirstnameComponent);
		this.COMPONENT_VALUE_MAP.set('lastname', LastnameComponent);
		this.COMPONENT_VALUE_MAP.set('company', BusinessNameComponent);
		this.COMPONENT_VALUE_MAP.set('address1', StreetComponent);
		this.COMPONENT_VALUE_MAP.set('zip', PostalcodeComponent);
		this.COMPONENT_VALUE_MAP.set('city', CityComponent);
		this.COMPONENT_VALUE_MAP.set('phone', PhoneComponent);
		this.COMPONENT_VALUE_MAP.set('dateOfBirth', DateofbirthComponent);
		this.COMPONENT_VALUE_MAP.set('gender', GenderComponent);
		this.COMPONENT_VALUE_MAP.set('zone', ZoneComponent);
		this.COMPONENT_VALUE_MAP.set('province', ZoneComponent);
		this.COMPONENT_VALUE_MAP.set('state', ZoneComponent);
		return ValueMapper.COMPONENT_VALUE_MAP;
	}
}
\end{lstlisting}

Im ersten Schritt wird, wenn das vom Template angeforderte Feld in der Map existiert, das Wrapper-Objekt erzeugt. 

Wenn dieses den Typ \texttt{ZoneComponent} besitzt, handelt es sich um ein RF, welches für ein Land wie Italien oder Amerika bestimmt ist. Um eine erfolgreiche Registrierung in diesen Ländern abzuschließen, muss hier nämlich die zugehörige Provinz (IT) oder der zugehöriger Bundesstaat (US) angegeben werden. Natürlich funktioniert das für jedes Land, welches dieses Kriterium erfüllen muss. Diese Daten werden anschließend aus dem \texttt{Country}-Objekt der Shopify-API gelesen und als \texttt{zone}-Property in die \texttt{options} gespeichert. 

Im Anschluss wird ebenfalls von diesem Objekt die Telefonnummernvorwahl in die \texttt{options} als \texttt{phoneNumberPrefix}-Property der Wrapper-Klasse gespeichert.

Als letzter Schritt in dieser Methode werden von Zeile 46 bis 48 die Labels, welche vorher in einer Map gespeichert wurden, als \texttt{label}-Property in die \texttt{options} der Wrapper-Objekte gespeichert. 

Nach dem Abschluss wird das Ergebnis an die Schleife der \texttt{initiateDynamicComponent-\\Configuration()}-Methode geliefert und als Zwischenergebnis gespeichert. Wenn die Iteration fertig ist, werden die Zwischenergebnisse zusammengefasst und an die nächste Methode übergeben.

\begin{figure}[h]
	\centerline{
		\includegraphics[width=1\textwidth, frame]{./grafiken/RF_Visualisierter Ablauf_1.png}
	}
	\vskip0pt
	\caption{Visualisierter Datenfluss vom Template bis zur Erstellung des 2D-Arrays}
\end{figure}

\subsection{\texttt{generateDynamicView(data: RegisterComponent[][])}}

\begin{lstlisting}[caption={Die \texttt{generateDynamicView()}-Methode}, language=JavaScript,label={lst:generateDynamicView}]
generateDynamicView(data: RegisterComponent[][]) {
for (const element of data) {
	const factory = this.componentFactoryResolver.
	resolveComponentFactory(AutoRowComponentGenerator);
	
	const ref = this.viewContainerRef.createComponent(factory);
	(<AutoRowComponentGenerator>ref.instance).data = element;
	ref.changeDetectorRef.detectChanges();
}
this.bindUserDataToViewData();
}
\end{lstlisting}

Die \texttt{generateDynamicView()}-Methode, markiert im Flussdiagramm~\ref{fig:fc} mit [3], iteriert über jedes Array, um eine \texttt{AutoRowComponentGenerator}-Klasse zu erzeugen. Aus jedem Array in dem 2D-Array wird also eine eigene Reihe für das RF erzeugt. Dabei wird die von Angular bereitgestellte \texttt{ComponentFactoryResolver}-Klasse verwendet, um aus einer Klasse einen Component zu erstellen. 
Jede Reiher, also jeder \texttt{AutoRowComponent-\\Generator}, wird dann nacheinander in die HTML-Datei durch eine \texttt{ViewContainerRef} eingefügt. Die \texttt{ViewContainerRef} verweist auf ein HTML-Element, wie es das Listing~\ref{lst:html} beschreibt.

\begin{lstlisting}[caption={ViewContainerRef verweist auf \#componentHook}, language=JavaScript,label={lst:html}]
<form #dynamicForm>
	<div #componentHook></div>
</form>
\end{lstlisting}

Das Array aus Wrapper-Klassen wird anschließend dem erzeugten Component als \texttt{data}-Objekt übergeben. \texttt{detectChanges()} sagt dem Component, dass er die Lifecycle-Hook-Methode \texttt{ngAfterViewInit()} erneut aufrufen muss. Wie im Listing~\ref{lst:nginARCG} erklärt wird, ist diese dafür verantwortlich, dass die übergebenen Daten verarbeitet werden.

Da das RF in der Profilübersicht wieder verwendet wird, ist es wichtig, die Daten, welche vom Benutzer bereits vorhanden sind, in das frisch erstellte RF durch die Methode \texttt{bindUserDataToViewData()} wieder einzubinden.

\subsection{Die \texttt{AutoRowComponentGenerator}-Klasse}

\begin{lstlisting}[caption={Die \texttt{ngAfterViewInit()}-Methode der \texttt{AutoRowComponentGenerator}-Klasse}, language=JavaScript,label={lst:nginARCG}]
ngAfterViewInit(): void {
	for (const x of this.data) {
		const factory = this.componentFactoryResolver
		.resolveComponentFactory(x.type);
		
		const ref = this.viewContainerRef.createComponent(factory);
		this.checkComponentRef(ref, x);
		ref.changeDetectorRef.detectChanges();
	}
	this.addClasses();
}
\end{lstlisting}

Anders als bei der \texttt{generateDynamicView()}-Methode wird nicht über ein 2D-Array iteriert, um ein Array an Daten zu übergeben, sondern über das tatsächliche Array, welches eine Reihe aus Feldern repräsentiert. Diese werden dann nach der Reihe auf die selbe Weise erzeugt. \texttt{checkComponentRef()} bindet die \texttt{options} der Wrapper-Klasse an die erzeugten Components, wie im Listing~\ref{lst:ccr} gezeigt wird. So ist es nun möglich, das RF dynamisch zur Laufzeit zu Erzeugen und in der UI anzuzeigen.

\begin{lstlisting}[caption={Die \texttt{checkComponentRef()}-Methode der \texttt{AutoRowComponentGenerator}-Klasse}, language=JavaScript,label={lst:ccr}]
checkComponentRef(ref: ComponentRef<any>, component: RegisterComponent) {
	if (component.options) {
		if (component.options.zones) {
			(ref.instance as ZoneComponent).zones = component.options.zones;
		}
		
		if (component.options.label) {
			(ref.instance as ComponentValueConfigurator).setLabel(
			component.options.label
			);
		}
		
		if (component.options.phonePrefix) {
			(ref.instance as PhoneComponent).setPhonePrefix(
			component.options.phonePrefix
			);
		}
	}
}
\end{lstlisting}

Abschließend werden noch CSS-Klassen für bestimmte Komponenten vergeben.

\begin{figure}[H]
	\centerline{
		\includegraphics[width=1\textwidth, frame]{./grafiken/RF_Visualisierter Ablauf_2.png}
	}
	\vskip0pt
	\caption{Visualisierter Datenfluss vom 2D-Array bis zum fertigen RF}
\end{figure}

\section{Die Validierungsmethodik}

\section{Die Komfortfunktionen}
\subsection{Automatische Adressvervollständigungsvorschläge}
\subsubsection{Der Google-API-Key}
Um auf die Google-API zugriff zu erhalten, wurde ein Google-API-Key erzeugt. Dieser wird aus folgenden Gründen erst bei der Verwendung im \texttt{AutocompleteComponent} geladen:

\begin{itemize}
	
	\item Sicherheit: Der API-Key wird nur dann als \texttt{script}-Element in den HTML-Head geladen, wenn dieser auch zur Verwendung gebraucht wird. Das dient dazu, dass der API-Key nicht sofort bei Hackerattacken oder von Webcrawler ausgelesen werden kann.
	
	\item Geschwindigkeit: Da diese Google-JS-Library keine Tree-Shakable-Library ist, sprich, man muss die komplette Bibliothek an Funktionen importieren, ist das Laden sehr zeitaufwändig. Je mehr \texttt{script}-Elemente im HTML-Head geladen werden, umso langsamer ladet die Website beim Aufruf.
	
	\item Preis: Je öfter der API-Key als \texttt{script}-Elemente im HTML-Head geladen wird, umso öfter wird von Google ein Preis für die Benutzung verrechnet.
\end{itemize}

Der API-Key wird über eine Direktive eines HTML-Elements in den HTML-Head geladen:

\begin{lstlisting}[caption={Die \texttt{ngOnInit()}-Methode der \texttt{LoadScriptDirective}}, language=JavaScript,label={lst:gpac}]
ngOnInit() {
	let node = document.createElement('script');
	node.src = this.param;
	node.type = 'text/javascript';
	node.async = false;
	let keyExists = false;
	let head = document.getElementsByTagName('head')[0];
	head.childNodes.forEach(x => {
		if (x.isEqualNode(node)) {
			keyExists = true;
		}
	});
	if (!keyExists) {
		head.appendChild(node);
	}
}
\end{lstlisting}

Der, auf in Zeile 3 verwiesene \texttt{this.param,} ist ein \texttt{@Input('script')}-Element, welches der API-Key übergeben wird. Wenn die \texttt{LoadScriptDirective} nicht mehr aktiv ist, wird das \texttt{script}-Element wieder aus dem DOM entfernt.

\subsubsection{Implementierung}
Die automatischen Adressvervollständigungsvorschläge kommen beim Laden des \texttt{Street-\\Component} zum Einsatz. Wenn der Benutzer im Feld \texttt{Straße und Hausnummer*} anfängt seine Adresse einzutippen, werden ihm automatische Vorschläge, basierend auf dem aktuell ausgewähltem Land, von der API präsentiert. 

\begin{figure}[H]
	\centerline{
		\includegraphics[width=1\textwidth, frame]{./grafiken/open_adress_completion.PNG}
	}
	\vskip0pt
	\caption{Automatisch generierte Adressvervollständigungsvorschläge in Österreich}
\end{figure}

\begin{figure}[H]
	\centerline{
		\includegraphics[width=1\textwidth, frame]{./grafiken/open_adress_completion_de.PNG}
	}
	\vskip0pt
	\caption{Automatisch generierte Adressvervollständigungsvorschläge in Deutschland}
\end{figure}

Dafür wurde eine Hierarchieebene darunter ein weiterer Komponent, der \texttt{Autocomplete-\\Component} eingebaut, welcher ein HTML-Input-Element mit der Direktive \texttt{\#addresstext} besitzt. Mithilfe des in Angular integriertem \texttt{@ViewChild('..')-Property decorator} ist es nun möglich, eine globales Objekt aus dem HTML-Input-Element zu erzeugen. 

\begin{lstlisting}[caption={Die \texttt{getPlaceAutocomplete()}-Methode der \texttt{AutocompleteComponent}-Klasse}, language=JavaScript,label={lst:gpac}]
getPlaceAutocomplete() {
	const autocomplete = new google.maps.places.Autocomplete(
	this.addresstext.nativeElement,
	{
		types: ['address']
	});
	
	google.maps.event.addListener(autocomplete, 'place_changed', () => {
		const place = autocomplete.getPlace();
		this.setAddress.emit(place);
	});
}
\end{lstlisting}

Wie im Listing~\ref{lst:gpac} ersichtlich ist, wird das Objekt anschließend dazu verwendet, um ein \texttt{google.maps.places.Autocomplete}-Objekt, welches als Parameter die zu generierenden Vorschläge übergeben wird, zu erzeugen. In diesem Fall wird dem Objekt gesagt, nur Adressvorschläge zu generieren. Auf dieses wird dann ein \texttt{place\_changed}-Event registriert. Wird das Event getriggert, feuert ebenfalls der \texttt{setAddress-\\EventEmitter} und liefert somit die übergebenen Daten an alle registrierten Observer.

\begin{lstlisting}[caption={Die \texttt{addressHasBeenSelected()}-Methode der \texttt{StreetComponent}-Klasse}, language=JavaScript,label={lst:gpac}]
addressHasBeenSelected(place: google.maps.places.PlaceResult) {
	let addressDto: AddressDto = {
		city: "",
		countryCode: "",
		postalcode: "",
		street: "",
		streetNumber: "",
	};
	
	place.address_components.forEach((x) => {
		if (x.types.includes("street_number")) {
			addressDto.streetNumber = x.long_name;
		} else if (x.types.includes("route")) {
			addressDto.street = x.long_name;
		} else if (x.types.includes("locality")) {
			addressDto.city = x.long_name;
		} else if (x.types.includes("postal_code")) {
			addressDto.postalcode = x.long_name;
		} else if (x.types.includes("country")) {
			addressDto.countryCode = x.short_name;
		}
	});
	
	this.handleAddressDto(addressDto);
}
\end{lstlisting}

Der \texttt{StreetComponent} hat sich mit der \texttt{addressHasBeenSelected()}-Methode auf dieses Event als Observer registriert. Da das überlieferte Objekt ein Array an Ergebnissen liefert, werden über diese iteriert und die Ergebnisse in ein temporäres Objekt gespeichert. Die \texttt{handleAddressDto()}-Methode aktualisiert jeden Wert in der \texttt{fields}-Map. 

Das interessante daran ist, dass, wenn manuell eine Adresse eines anderen Landes eingegeben wird, ein Event im \texttt{ValidatorService}, auf welches der \texttt{UserprofileComponent} registriert ist, abgefeuert wird. Es wird umgehend die \texttt{onInitViewChange()}-Methode aufgerufen, welche alle Nachfolgenden Elemente des HTML-Form-Elements, zu finden im Listing~\ref{lst:html}, löscht und wieder bei der \texttt{initiateDynamicComponentConfiguration()}-Methode startet. Somit wird ein, für das neue Land angepasste, RF generiert (Kapitel ~\ref{ssec:lblInitDCC}).


\subsection{Ermittlung des Standorts durch die IP - Use my location}

\subsection{Automatische Telefonnummerprefix Erkennung}

Wie im Kapitel~\ref{ssec:lblAnalyzeRow} [\texttt{analyzeRow()}] bereits erklärt wird, wird die Telefonnummernvorwahl aus dem \texttt{Country}-Objekt der Shopify-API gelesen und in die \texttt{options} als \texttt{phoneNumberPrefix}-Property der Wrapper-Klasse gespeichert.
	
	%%%%%%%%%%%%%%%%%%%%%%%%%%%%%%%%%%%%%%%%%
	%%%   CI / CD   %%%%%%%%%%%%%%%%%%%%%%%%%
	%%%%%%%%%%%%%%%%%%%%%%%%%%%%%%%%%%%%%%%%%
	
	\chapter{CI / CD}
\section{Was ist CI/CD}

Es handelt sich hier um Methode, bei der den Kunden regelmäßig Apps bereitgestellt und alle Phasen der Anwendungsentwicklung automatisiert werden. Die Hauptkonzepte von CI/CD sind Continuous Integration, Continuous Delivery und Continuous Deployment. CI/CD löst die Probleme, welche die Integration von neuem Code für DevOps-Teams verursachen kann.\cite{whatIsCICD}

Insbesondere sorgt CI/CD für eine kontinuierliche Automatisierung und Überwachung über den gesamten App-Lifecycle hinweg, von der Integrations- und Test- bis hin zur Bereitstellungs- und Implementierungsphase. Diese zusammenhängenden Praktiken werden oft als „CI/CD-Pipeline“ bezeichnet, und sie werden durch eine agile Zusammenarbeit der DevOps-Teams unterstützt.\autocite{whatIsCICD}

Die Abkürzung CI/CD hat unterschiedliche Bedeutungen. „CI“ bedeutet Continuous Integration, also der Automatisierungsprozess für Entwickler. Bei einer erfolgreichen CI werden regelmäßig neue Codeänderungen für Apps entwickelt, geprüft und in einem gemeinsamen Repository zusammengeführt. Damit soll der Konflikt verhindert werden, den zu viele Branches einer App verursachen können, wenn sie zeitgleich entwickelt werden.\autocite{whatIsCICD}

„CD“ bedeutet Continuous Delivery bzw. Continuous Deployment. Dass sind verwandte Konzepte, die zuweilen synonym verwendet werden. Obwohl es bei beiden Konzepten um die Automatisierung weiterer Phasen der Pipeline geht, werden die Begriffe manchmal unterschiedlich verwendet, um das Ausmaß der Automatisierung zu verdeutlichen.\autocite{whatIsCICD}

Continuous Delivery bedeutet üblicherweise, dass App-Änderungen eines Entwicklers automatisch auf Bugs getestet und in ein Repository (wie GitHub oder eine Container-Registry) hochgeladen werden, von wo aus sie vom Operations-Team in einer Live-Produktivumgebung bereitgestellt werden können. Dieser Vorgang ist die Antwort auf Transparenz- und Kommunikationsprobleme zwischen Dev- und Business-Teams. Damit soll sichergestellt werden, dass neuer Code mit minimalem Aufwand implementiert werden kann.\autocite{whatIsCICD}

Continuous Deployment (das andere „CD“) kann sich auf die automatische Freigabe von Entwickleränderungen vom Repository zur Produktivphase beziehen, wo sie direkt vom Kunden genutzt werden können. Dieser Vorgang soll der Überlastung von Operations-Teams bei manuellen Prozessen entgegenwirken, die die Anwendungsbereitstellung verlangsamen. Continuous Development baut die Vorteile der Continuous Delivery aus, indem auch noch die nächste Phase der Pipeline automatisiert wird.\autocite{whatIsCICD}

\begin{figure}[h]
	\centerline{
		\includegraphics[width=0.8\textwidth]{./grafiken/ci-cd-flow-redhatsource.png}
	}
	\vskip0pt
	\caption{CI/CD Workflow (Bild wurde entnommen aus \cite{whatIsCICD})}
\end{figure}


Manchmal sind mit CI/CD lediglich die zusammenhängenden Praktiken der Continuous Integration und der Continuous Delivery, manchmal aber auch alle drei Konzepte der Continuous Integration, Continuous Delivery und Continuous Deployment gemeint. Noch komplizierter wird das Ganze dadurch, dass mit Continuous Delivery zuweilen auch die Prozesse des Continuous Deployment mitgemeint sind.\autocite{whatIsCICD}

Letztendlich bringen uns diese Details jedoch nicht weiter. Sehen Sie CI/CD einfach als Prozess an, der nicht selten als Pipeline visualisiert wird und der ein hohes Maß an kontinuierlicher Automatisierung und Überwachung bei der Anwendungsentwicklung umfasst. Je nach Fall hängt die Auslegung des Begriffs vom Grad der Automatisierung der CI/CD-Pipeline ab. Viele Unternehmen arbeiten zunächst mit CI und setzen den Prozess später mit der automatischen Bereitstellung und Implementierung fort, z. B. bei cloudnativen Apps.\autocite{whatIsCICD}

\section{Continuous Integration}

Bei der modernen Anwendungsentwicklung arbeiten mehrere Entwickler an unterschiedlichen Features der gleichen App. Die gleichzeitige Zusammenführung aller Quellcode-Branches an einem Tag (auch bekannt als „Merge Day“) kann einen hohen Arbeits- und Zeitaufwand bedeuten. Der Grund dafür ist, dass Anwendungsänderungen von getrennt arbeitenden Entwicklern miteinander in Konflikt treten können, wenn sie zeitgleich durchgeführt werden. Dieses Problem kann sich verschlimmern, wenn jeder Entwickler seine eigene lokale Integrated Development Environment (IDE) definiert, statt im Team eine gemeinsame cloudbasierte IDE zu erstellen.\autocite{whatIsCICD}

Mithilfe der Continuous Integration (CI) können Entwickler ihre Codeänderungen in einem gemeinsamen „Branch“ oder „Trunk“ der Anwendung viel häufiger zusammenführen, manchmal sogar täglich. Sobald die Änderungen eines Entwicklers zusammengeführt werden, werden sie in automatischen App-Builds und unterschiedlichen Stufen von Automatisierungsprüfungen (normalerweise Einheits- und Integrationstests) validiert. So wird sichergestellt, dass die Funktionsfähigkeit nicht beeinträchtigt wurde. Dabei müssen alle Klassen und Funktionen bis hin zu den verschiedenen Modulen der App getestet werden. Wenn die automatische Prüfung Konflikte zwischen aktuellem und neuem Code erkennt, lassen sich diese mithilfe von CI schneller und häufiger beheben.\autocite{whatIsCICD}

\section{Continuous Delivery}

Nach der Automatisierung von Builds und Einheits- und Integrationstests bei der CI wird bei der Continuous Delivery auch die Freigabe des validierten Codes an ein Repository automatisch durchgeführt. Um also einen effizienten Continuous Delivery-Prozess zu gewährleisten, muss die CI bereits in Ihre Entwicklungs-Pipeline integriert sein. Ziel der Continuous Delivery ist eine Codebasis, die jederzeit in einer Produktivumgebung bereitgestellt werden kann.\autocite{whatIsCICD}

Bei der Continuous Delivery umfasst jede Phase − von der Zusammenführung der Codeänderungen bis zur Bereitstellung produktionsreifer Builds − automatisierte Tests und Code-Freigaben. Am Ende dieses Prozesses kann das Operations-Team eine App schnell und einfach in der Produktivphase bereitstellen.\autocite{whatIsCICD}

\section{Continuous Deployment}

Die abschließende Phase der CI/CD-Pipeline ist das Continuous Deployment. Als Erweiterung der Continuous Delivery, bei der produktionsreife Builds automatisch an ein Code-Repository freigegeben werden, wird beim Continuous Deployment auch die Freigabe einer App in die Produktivphase automatisiert. Da der Produktivphase in der Pipeline kein manuelles Gate vorgeschaltet ist, müssen beim Continuous Deployment die automatisierten Tests immer sehr gut durchdacht sein.\autocite{whatIsCICD}

In der Praxis bedeutet Continuous Deployment, dass App-Änderungen eines Entwicklers binnen weniger Minuten nach ihrer Erstellung live gehen können (vorausgesetzt, sie bestehen den automatischen Test). Dies erleichtert eine kontinuierliche Integration von User Feedback ungemein. All diese zusammenhängenden CI/CD-Praktiken machen eine Anwendungsimplementierung weniger riskant, weil Änderungen in Teilen und nicht auf einmal freigegeben werden. Die Vorabinvestitionen sind allerdings beträchtlich, da automatische Tests für die diversen Prüf- und Release-Phasen in der CI/CD-Pipeline geschrieben werden müssen.\autocite{whatIsCICD}

\section{GitLab Pipeline}

Pipelines sind top-level Komponenten der Continuous Integration, Delivery und Deployment. Pipelines umfassen:\autocite{gitlabPipelines}

\begin{itemize}
	\item Jobs, welche definieren \textit{was} auszuführen ist. Zum Beispiel gibt es Jobs, welche den Code kompilieren, Packete installieren und testen.
	\item Stages, welche definieren \textit{wann} Jobs auszuführen sind. Beispielsweise ist die Kompile-Stage üblicherweise vor der Test-Stage.
\end{itemize}

Jobs werden von Runner ausgeführt. Mehrere Jobs in der selben Stage werden parallel ausgeführt, wenn genügend simultane Runner bereitstehen.
Wenn \textit{alle} Jobs in einer Stage erfolgreich ausgeführt wurden, geht die Pipeline zur nächsten Stage über.
Wenn \textit{irgendein} Job in einer Stage fehlschlägt, wird die nächste Stage (normalerweise) nicht ausgeführt und die Pipeline endet vorzeitig.\autocite{gitlabPipelines}

\newpage

Im Allgemeinen werden Pipelines in der Regel automatisch  nach dem Hochladen eines Commits ausgeführt und erfordern nach ihrer Erstellung keinen weiteren manuellen Eingriff.
Eine typische Pipeline kann aus vier Phasen bestehen, die in der folgenden Reihenfolge ausgeführt werden:\autocite{gitlabPipelines}

\begin{itemize}
	\item Eine \colorbox{lightgray}{\texttt{Build-Stage}}, mit einem \colorbox{lightgray}{\texttt{Compile-Job}}.
	\item Eine \colorbox{lightgray}{\texttt{Test-Stage}}, welche mehrere \colorbox{lightgray}{\texttt{Test-Jobs}} beinhalten kann.
	\item Eine \colorbox{lightgray}{\texttt{Staging-Stage}}, mit einem \colorbox{lightgray}{\texttt{Deploy-to-Stage-Job}}.
	\item Eine \colorbox{lightgray}{\texttt{Production-Stage}}, mit einem \colorbox{lightgray}{\texttt{Deploy-to-Production-Job}}.
\end{itemize}

\section{GitLab Runner}
\subsection{Registrierung}

GitLab Runner ist eine Anwendung, die mit GitLab CI/CD arbeitet, um Aufträge in einer Pipeline auszuführen.
Man kann die GitLab Runner-Anwendung auf der Infrastruktur installieren, die man besitzen oder verwalten. In diesem Fall sollte man GitLab Runner auf einem Rechner installieren, der von dem Rechner getrennt ist, der die GitLab-Instanz hostet. GitLab Runner ist Open-Source und in Go geschrieben. Er kann als einzelne Binärdatei ausgeführt werden, wobei es keine sprachspezifischen Anforderungen benötigt.
Es ist auch möglich, GitLab Runner auf verschiedenen unterstützten Betriebssystemen installieren. Andere Betriebssysteme sind ebenfalls in der Lage Runner bereitzustellen, sofern diese eine Go-Binärdatei kompilieren können.
GitLab Runner kann auch in einem Docker-Container ausgeführt oder in einem Kubernetes-Cluster bereitgestellt werden.\autocite{gitlabRunner}

\begin{figure}[h]
	\centerline{
		\includegraphics[width=1.1\textwidth]{./grafiken/gitlab_runner_status.JPG}
	}
	\vskip0pt
	\caption{GitLab Runner Instanz auf einem lokalen Windows-Rechner}
\end{figure}

\subsection{Executors}

Wenn Sie einen Runner registrieren, müssen Sie einen Executor auswählen. Ein Executor bestimmt die Umgebung, in der jeder Job läuft.\autocite{gitlabRunner}

Wenn man zum Beispiel will, dass ein CI/CD-Auftrag PowerShell-Befehle ausführt, kann man GitLab Runner auf einem Windows-Server installieren und dann einen Runner registrieren, der den Shell-Executor verwendet.
Wenn man möchte, dass ein CI/CD-Job Befehle in einem benutzerdefinierten Docker-Container ausgeführt wird, muss man GitLab Runner auf einem Linux-Server installieren und einen Runner registrieren, der den Docker-Executor verwendet.
Dies sind nur einige der möglichen Konfigurationen. Man könnte GitLab Runner auch auf einer virtuellen Maschine installieren und eine andere virtuelle Maschine als Executor verwenden lassen.
Wenn man GitLab Runner in einem Docker-Container installieren und den Docker-Executor für die Ausführung der Jobs auswählen, wird dies manchmal als "Docker-in-Docker"-Konfiguration bezeichnet.\autocite{gitlabRunner}

\subsection{Shared Runners}

Shared Runners sind für jedes Projekt in einer GitLab-Instanz verfügbar.

Shared Runners werden verwendet, wenn man mehrere Jobs mit ähnlichen Anforderungen in mehreren Projekten hat. Also anstatt viele Runner für viele Projekte im Leerlauf zu haben, kann man eine Gruppe von Runner für alle Projekte benützen.\autocite{gitlabSharedRunner}

\subsection{Tags}

Registrierte Runner kann man sogenannte Tags zuweisen. Wenn ein CI/CD Job läuft, kann dieser anhand seiner zugewiesener Tags determinieren, welchen Runner er verwenden soll. Das hat den Vorteil, dass man für beispielsweise einen Job, der ein Ruby Projekt kompilieren soll, seinen Runner nicht umkonfigurieren muss, sondern automatisch der Ruby-Runner verwendet wird.

Man muss dazu lediglich folgendes in seine  \colorbox{lightgray}{\texttt{.gitlab-ci.yml}} Datei einbinden:

\begin{figure}[h]
	\centerline{
		\includegraphics{./grafiken/ruby_runner_tag_in_gitlab-ci-yml_file.JPG}
	}
	\vskip0pt
	\caption{ruby Tag in gitlab-ci.yml Datei (Bild wurde entnommen aus \autocite{gitlabRunner})}
\end{figure}
\autocite{gitlabPipelines}

\begin{lstlisting}[caption={yaml besipiel},captionpos=b, numbers=left, backgroundcolor=\color{black!10}, language=docker-compose]
	---
	key: value
	map:
	key1: "foo:bar"
	key2: value2
	list:
	- element1
	- element2
	# This is a comment
	listOfMaps:
	- key1: value1a
	key2: value1b
	- key1: value2a
	key2: value2b
	---
\end{lstlisting}

	
	%%%%%%%%%%%%%%%%%%%%%%%%%%%%%%%%%%%%%%%%%
	%%%   BESCHREIBUNG AUS ANWENDERSICHT   %%
	%%%%%%%%%%%%%%%%%%%%%%%%%%%%%%%%%%%%%%%%%
	
	\chapter{Beschreibung aus Anwendersicht} \label{anwendersicht}

Dieses Projekt ist für den Kunden entwickelt worden, daher wird hier die Anwendersicht beschrieben.

\section{Länderwahl}
\begin{figure}[h]
	\centerline{
		\includegraphics[width=1\textwidth, frame]{./grafiken/erm_country_selection.png}
	}
	\vskip0pt
	\caption{Screenshot von Länderwahl} \label{fig:countrySelection}
\end{figure}
Um auf die Webseite zu gelangen, ist es Erforderlicht, dass man davor in der Länderwahl, wie in Abbildung \ref{fig:countrySelection} sein Land und bei den Ländern, wo es mehrere Landssprachen gibt, auch die Sprache wählt. Aus dem Länder- und Sprachenkürzel setzt sich die Locale zusammen. Für Österreich wäre die Locale daher "de\_AT" und für Deutschland "de\_DE". Diese Locale wird in den Cookies gespeichert und die Webseite richtet sich nach dieser Locale in den Cookies. Es ist auch möglich, im Nachhinein das Land und die Sprache zu ändern, indem man im Header der Webseite auf das Land klickt. Durch diesen Klick gelangt man von überall auf die Länderwahl.

\section{Startseite}
\subsection{Startseite vor dem Login}
\begin{figure}[H]
	\centerline{
		\includegraphics[width=1\textwidth, frame]{./grafiken/erm_home_not_logged_in_1.png}
	}
	\vskip0pt
	\caption{Screenshot von der Startseite nicht eingeloggt} \label{fig:homeNotLoggedIn}
\end{figure}

Nach der Länderwahl gelangt man zu der Startseite. Anfangs ist man noch nicht eingeloggt, daher ist ein Login-Feld in der Mitte, wie man in Abbildung \ref{fig:homeNotLoggedIn} sehen kann. Da sieht man auch, dass das Feld "Mein Maschinenpark" etwas heller ist als die anderen Felder, da man den Maschinenpark nur als eingeloggter Benutzer nutzen kann. Die zwei anderen Felder sind auch ohne Registrierung nutzbar.
 
\subsection{Startseite nach einem erfolgreichen Login}
\begin{figure}[H]
	\centerline{
		\includegraphics[width=1\textwidth, frame]{./grafiken/erm_home_logged_in.png}
	}
	\vskip0pt
	\caption{Screenshot von der Startseite eingeloggt} \label{fig:homeLoggedIn}
\end{figure}

Nach einem erfolgreichen Login ist das Feld "Mein Maschinenpark" aktiv. Als eingeloggter Benutzer hat man nun Zugriff auf den Maschinenpark. Das erkennt man, da das Feld in Abbildung \ref{fig:homeLoggedIn} den selben Grauton hat, als die beiden anderen Felder. 

\section{Registrierungsseite}

Falls man noch nicht registriert ist und man auf der Startseite auf "Registrieren" drückt, wird man zur Registrierungsseite weitergeleitet. Den ersten Teil dieser Seite sieht man in Abbildung \ref{fig:register}

\subsection{Registrierung}
\begin{figure}[H]
	\centerline{
		\includegraphics[width=1\textwidth, frame]{./grafiken/erm_register.png}
	}
	\vskip0pt
	\caption{Screenshot von der Registrierungsseite} \label{fig:register}
\end{figure}

Während der Passworteingabe bekommt man Feedback, wie sicher das ausgewählte Passwort ist und was noch fehlt, um den Passwortanforderungen zu entsprechen. Die Passwortanforderungen, die man in Abbildung \ref{fig:pwHints} sieht, entsprechen den Voraussetzungen von Auth0, da die Registrierung über Auth0 läuft. Der Balken unter dem Passwortfeld, wie in Abbildung \ref{fig:pwSec} gibt die Sicherheit des Passwortes an. Das Passwort wird mit gängigen Namen und Nachnamen aus Amerika, beliebte englische Wörter und Muster wie Datumsangaben, Wiederholungen und Tastaturmuster verglichen um die Stärke angeben zu können.

\begin{figure}[H]
	\centerline{
		\includegraphics[width=1\textwidth, frame]{./grafiken/passwordSecurity.PNG}
	}
	\caption{Screenshot von der Passwortsicherheit} \label{fig:pwSec}
\end{figure}
\begin{figure}
	\centerline{
	\includegraphics[width=1\textwidth, frame]{./grafiken/passwordHints.PNG}
	}	
	\caption{Screenshot von den Passworthinweisen} \label{fig:pwHints}
\end{figure}

\subsection{Weitere Daten für die Registrierung}

In diesem Schritt der Registrierung füllt man persönliche Daten aus. Das Land und die Sprache werden von der am Beginn ausgewählten Locale übernommen. Diese kann man aber auch hier noch ändern. Wenn das Land oder die Sprache geändert wird, wird dieses Formular neu generiert und die Felder werden nach den Standard des Landes erzeugt. Das bedeutet, dass die Reihenfolge der angezeigten Felder von dem ausgewählten Land abhängt, da in gewissen Ländern der Nachname vor dem Vornamen in einem Formular steht oder es in bestimmten Ländern Staaten oder Provinzen dazu kommen. Die Feldbeschreibung wird auch nach der ausgewählten Sprache geändert. Weiters wird auch die Validierung der Felder angepasst, denn zum Beispiel die Postleitzahl in den verschiedenen Länder variiert (4-stellig in Österreich, 5-stellig in Deutschland). \\
Für die Adresseingabe in der Abbildung \ref{fig:register2} hat der Benutzer 3 Möglichkeiten:\\
Es besteht die Möglichkeit, jedes Adressfeld selber auszufüllen. Somit sind die Felder Straße und Hausnummer, Ort, Postleitzahl und bei gewissen Ländern die Provinzen oder Staaten auszufüllen.\\
Eine etwas schnellere Variante ist die Verwendung von dem integrierten Typeaheads. Bei der Eingabe des Feldes Straße und Hausnummer erscheinen Adresseingabevorschläge darunter. Wenn man auf die gewünschte Adresse klickt, werden die Felder Ort, Postleitzahl und falls vorhanden die Provinz oder der Staat automatisch ausgefüllt. Alle Vorschläge, die angezeigt werden, sind in den ausgewählten Land. Somit wird verhindert, dass der Benutzer eine Adresse eingibt, die sich nicht in dem gewählten Land befindet.\\
Die schnellste Variante ist die Verwendung von der integrierten Standorterkennung. Dafür drückt man den "Use my location" über dem Adressfeld. Dadurch wird der Standort bestmöglich erkannt und alle vorhandenen Adressfelder werden automatisch ausgefüllt. Falls der Standort nicht ganz richtig erkannt wurde, kann man es danach noch bearbeiten. Die Standorterkennung ändert auch das ausgewählte Land, falls diese nicht übereinstimmen.
\begin{figure}[H]
	\centerline{
		\includegraphics[width=1\textwidth, frame]{./grafiken/erm_register_data.png}
	}
	\vskip0pt
	\caption{Screenshot von der Dateneingabe der Registrierung} \label{fig:register2}
\end{figure}

Um anzugeben, ob der Benutzer auf einem landwirtschaftlichen Betreib arbeitet, besteht die Möglichkeit, dass er auf dem eigenen landwirtschaftlichen Betrieb arbeitet, auf einem anderen landwirtschaftlichen Betrieb angestellt ist oder dass er auf keinen Betrieb arbeitet. Falls der Benutzer auswählt, dass er auf einen anderen Betrieb angestellt ist, erscheint zusätzlich das Feld für den Namen des Unternehmens.\\
Jedes Eingabefeld wird sofort nach Abgabe des Cursorfokus validiert. Falls ein Fehler auftritt wird das Feld rot umrandet und unter dem Feld erscheint eine Fehlermeldung, was falsch ist. Die Eingaben werden im Frontend und im Backend validiert. In der Abbildung \ref{fig:eingabeError} sieht man die Felder mit einigen möglichen Fehlermeldungen. Bei einer korrekten Eingabe, verschwindet der rote Rahmen um das jeweilige Feld wieder.
\begin{figure}[H]
	\centerline{
		\includegraphics[width=1\textwidth, frame]{./grafiken/dateneingabe_Errors.PNG}
	}
	\vskip0pt
	\caption{Screenshot von den Fehlermeldungen der Eingaben} \label{fig:eingabeError}
\end{figure}

\subsection{Letzter Schritt der Registrierung}
Um die Registrierung erfolgreich abzuschließen, bekommt man eine Bestätigung per E-Mail zugesendet. Darin befindet sich ein Aktivierungslink. Den letzte Schritt der Registrierung sieht man in der Abbildung \ref{fig:step3register}. Hier wird man nur mehr gebeten, den Link in der E-Mail zu klicken. Nach der Aktivierung ist die Webseite ohne Einschränkungen nutzbar.
\begin{figure}[H]
	\centerline{
		\includegraphics[width=1\textwidth, frame]{./grafiken/erm_register_final.png}
	}
	\vskip0pt
	\caption{Screenshot von dem letzten Schritt der Registrierung} \label{fig:step3register}
\end{figure}

Nach der Bestätigung in der E-Mail wird man zurück auf die Startseite geleitet und es erscheint ein Pop-Up, wie man in Abbildung \ref{fig:popup} sieht.

\begin{figure}[H]
	\centerline{
		\includegraphics[width=1\textwidth, frame]{./grafiken/erm_home_after_email.png}
	}
	\vskip0pt
	\caption{Screenshot von dem Bestätigungs-Pop-Up} \label{fig:popup}
\end{figure}

\section{Profilübersicht}

Auf die Profilübersicht gelangt man, wenn man in dem Bestätigungs-Pop-Up auf "Profil bearbeiten" klickt oder man klickt auf den eigenen Namen rechts oben und danach auf "Mein Profil". Die Profilübersicht sieht man in Abbildung \ref{fig:profil}.

\begin{figure}[H]
	\centerline{
		\includegraphics[width=1\textwidth, frame]{./grafiken/erm_profil.png}
	}
	\vskip0pt
	\caption{Screenshot von der Profilübersicht} \label{fig:profil}
\end{figure}

In der Profilübersicht kann man all seine persönlichen Daten einsehen und diese auch bearbeiten. Es besteht auch die Möglichkeit, die E-Mail und das Passwort zu ändern.

\subsection{Persönliche/Betriebliche Daten}

Nach der Registrierung ist es möglich, hier noch weiter persönliche und betriebliche Daten einzutragen. Falls der Benutzer ausgewählt hat, dass er auf dem eigenen oder einem anderen landwirtschaftlichen Betrieb arbeitet, erweitern sich die Daten um einige Felder für den Betrieb. Bei einem Klick auf "Persönliche/Betriebliche Daten" klappt sich ein Datenformular auf, dass man in Abbildung \ref{fig:profilData} sehen kann.

\begin{figure}[H]
	\centerline{
		\includegraphics[width=1\textwidth, frame]{./grafiken/erm_profil_daten.png}
	}
	\vskip0pt
	\caption{Persönliche/Betriebliche Daten} \label{fig:profilData}
\end{figure}

Bei den Daten über den Betrieb wird gefragt, wie der Benutzer auf den Betrieb angestellt ist. Weiters ist auszufüllen, um welche Art von landwirtschaftlichen Betrieb es sich handelt. Die Betriebsfläche ist in Hektar und die Ackerfläche in Prozent anzugeben und für die Bewirtschaftungsform stehen Konventionell und Biologisch zur Auswahl. Danach wählt man noch die Betriebsform und die Anzahl der jeweiligen Tiere.

\subsection{Newsletter}

Die Firma Pöttinger bietet Newsletter in verschiedenen Sprachen für die Kunden an. Es besteht die Möglichkeit sich von diesen Newsletter jederzeit an- oder abzumelden. Das geschieht mit einem Klick auf den jeweiligen Knopf unter der gewünschten Sprache. Falls ein Newsletter abonniert ist, sieht man darunter, seit wann man diesen Newsletter erhält.

\begin{figure}[H]
	\centerline{
		\includegraphics[width=1\textwidth, frame]{./grafiken/erm_profil_newsletter.png}
	}
	\vskip0pt
	\caption{Newsletter} \label{fig:newsletter}
\end{figure}

\section{Produktpalette}

Mit der Produktpalette erhält man eine Vielzahl an Informationen zu den Landmaschinen der Firma Pöttinger. Um die gewünschte Maschine zu finden, gibt es verschiedene Möglichkeiten. Eine Möglichkeit ist, die Maschine in der Produktpalette zu suchen. Dafür klickt man auf "Suche in der Produktpalette" und danach öffnet sich ein Baumstruktur, in der man von oben nach unten in den Kategorien suchen kann. Am Beginn stehen Grünland und Ackerbau zur Auswahl. Öffnet man Grünland, erscheinen die Unterkategorien dieser Kategorie. Es gibt bis zu fünf Ebenen, bis man zu den einzelnen Maschinen gelangt. Diese Suche in der Baumstruktur kann man mit einer Eingabe in der Stichwortsuche über dem Baum noch einschränken. Wenn das eingegebene Stichwort in einer Ebene und darunter nicht vorkommt, sieht man diese Ebene auch nicht mehr. In Abbildung \ref{fig:produktpalette} sieht man die geschlossene Produktpalette. In der Abbildung \ref{fig:produktpaletteOffen} sieht man ein Beispiel für Produktpalette mit offenen Unterebenen und in Abbildung \ref{fig:produktpaletteMitStichwort} sieht man den Unterschied, wenn man die offene Produktpalette mit einer Stichworteingabe verbindet.

\begin{figure}[H]
	\centerline{
		\includegraphics[width=1\textwidth, frame]{./grafiken/erm_produktpalette.png}
	}
	\vskip0pt
	\caption{Produktpalette} \label{fig:produktpalette}
\end{figure}

\begin{figure}[H]
	\centerline{
		\includegraphics[width=1\textwidth, frame]{./grafiken/erm_produktpalette_offen.png}
	}
	\vskip0pt
	\caption{Offene Produktpalette} \label{fig:produktpaletteOffen}
\end{figure}

\begin{figure}[H]
	\centerline{
		\includegraphics[width=1\textwidth, frame]{./grafiken/erm_produktpalette_offen_stichwort.PNG}
	}
	\vskip0pt
	\caption{Offene Produktpalette mit Stichwortsuche} \label{fig:produktpaletteMitStichwort}
\end{figure}

Eine andere Möglichkeit ist noch die Suche mit einer gültigen Maschinennummer. Gibt man diese Nummer ein, gelangt man direkt zur Produktübersicht.

\section{Produktübersicht}

Hat man sich eine Maschine in der Produktpalette, direkt mit der Maschinennummer oder eine in Maschinenpark gespeicherte Maschine gesucht, gelangt man auf die Produktübersicht, die man in Abbildung \ref{fig:produktübersicht} sehen kann. Hier findet man jegliche nützliche Information der ausgewählten Maschine. Die angezeigten Informationen variieren je nach Maschine und Verfügbarkeit.

\begin{figure}[H]
	\centerline{
		\includegraphics[width=1\textwidth, frame]{./grafiken/erm_produktübersicht.png}
	}
	\vskip0pt
	\caption{Produktübersicht} \label{fig:produktübersicht}
\end{figure}

\subsection{Highlights}

Bei den Highlights des Produktes werden wissenswerte Informationen in Form eines Bildes mit einer Überschrift aufgelistet. Das sieht man in Abbildung \ref{fig:highlight}.
\begin{figure}[H]
	\centerline{
		\includegraphics[width=1\textwidth, frame]{./grafiken/erm_detailansicht_highlights.PNG}
	}
	\vskip0pt
	\caption{Highlights} \label{fig:highlight}
\end{figure}
 Wird nun ein Highlight ausgewählt, gelangt man zu einer genaueren Beschreibung der jeweiligen Besonderheit. So einen Artikel sieht man in Abbildung \ref{fig:highlightbeschreibung}.
\begin{figure}[H]
	\centerline{
		\includegraphics[width=1\textwidth, frame]{./grafiken/erm_detailansicht_highlights_beschreibung.PNG}
	}
	\vskip0pt
	\caption{Highlight-Beschreibung} \label{fig:highlightbeschreibung}
\end{figure}

\subsection{Ausstattung}

Bei der Ausstattung findet man die verbauten Teile. Wie in Abbildung \ref{fig:ausstattung} sieht man diese Teile mit der jeweiligen Materialnummer und einer Bezeichnung aufgelistet. Darunter sieht man noch optionale Ausrüstungen, die mit dieser Maschine kompatibel sind.

\begin{figure}[H]
	\centerline{
		\includegraphics[width=1\textwidth, frame]{./grafiken/erm_detailansicht_ausstattung.PNG}
	}
	\vskip0pt
	\caption{Ausstattung} \label{fig:ausstattung}
\end{figure}

\subsection{Technische Daten}

Bei den technischen Daten in Abbildung \ref{fig:technischeDaten} findet man nützliche Informationen wie Anbau, Gewicht, Arbeitsbreite, Transportbreite, Kraftbedarf und vieles mehr.

\begin{figure}[H]
	\centerline{
		\includegraphics[width=1\textwidth, frame]{./grafiken/erm_detailansicht_technisch.PNG}
	}
	\vskip0pt
	\caption{Ausstattung} \label{fig:technischeDaten}
\end{figure}
\subsection{Videos, Bilder, Prospekte und Betriebsanleitung}
In diesen Kategorien findet man Videos, Bilder, Prospekte und eine Betriebsanleitung für die Maschine. Diese Videos werden über YouTube abgespielt. Die Prospekte und die Betriebsanleitung sind zum Download verfügbar. Da nicht von jeder Maschine Bilder oder Videos vorhanden sind, Gibt es dieses Feld auch nicht bei jeder Detailansicht.
\section{Maschinenpark}

In Abbildung \ref{fig:maschinenpark} sieht man den Maschinenpark, in dem eine Auflistung aller gespeicherten Maschinen des Benutzers ist. Somit hat der Nutzer einen schnellen Zugriff auf nützliche Tipps zur der gespeicherten Maschine, Bedienungsanleitungen, Ersatzteillisten, Wartungsinformationen, sowie alle technischen Details und Unterlagen.

\begin{figure}[H]
	\centerline{
		\includegraphics[width=1\textwidth, frame]{./grafiken/erm_maschinenpark.png}
	}
	\vskip0pt
	\caption{Maschinenpark} \label{fig:maschinenpark}
\end{figure}

Navigiert man über den Maschinenpark zu der gespeicherten Maschine, ändert sich die Detailansicht, wie in Abbildung \ref{fig:savedMaschine}, etwas. Hinzugekommen ist ein Bild der Maschine und es bietet die Möglichkeit, diese Maschine einen individuellen Namen zu vergeben.

\begin{figure}[H]
	\centerline{
		\includegraphics[width=1\textwidth, frame]{./grafiken/erm_detailansicht_saved_machine.png}
	}
	\vskip0pt
	\caption{Detailansicht der gespeicherten Maschine} \label{fig:savedMaschine}
\end{figure}

Wenn die Maschine einen individuellen Namen erhalten hat, wird dieser Name auch im eigenen Maschinenpark wie in Abbildung \ref{fig:benannteMaschine} angezeigt:
 
\begin{figure}[H]
	\centerline{
		\includegraphics[width=1\textwidth, frame]{./grafiken/erm_maschinenpark_benannteMaschine.png}
	}
	\vskip0pt
	\caption{Benannte Maschine} \label{fig:benannteMaschine}
\end{figure}
	
	%%%%%%%%%%%%%%%%%%%%%%%%%%%%%%%%%%%%%%%%%
	%%%   DANKSAGUNG   %%%%%%%%%%%%%%%%%%%%%%
	%%%%%%%%%%%%%%%%%%%%%%%%%%%%%%%%%%%%%%%%%
	
	\chapter{Danksagung}
Als erstes möchten wir einen besonderen Dank an unsere Eltern, Franz und Renate Schörgendorfer und Monika Hörzi, aussprechen, welche es uns ermöglicht haben, die HTBLA Grieskirchen zu besuchen und welche uns in jeder Phase dieser Zeit mit allen Mitteln unterstützt haben.\\

Wir möchten uns auch sehr herzlich beim Auftraggeber unserer Diplomarbeit, dem Unternehmen \ThPartnerName \,, für das Vertrauen und die Bereitstellung der Diplomarbeit bedanken.\\

Ebenfalls bedanken wir uns bei allen Professorinnen und Professoren, die uns bei technischen und organisatorischen Fragen ihre Hilfen angeboten haben. Besonders möchten wir unseren Dank gegenüber unserem Diplomarbeitsbetreuer \ThSupervisorName \, aussprechen.
	
	%%%%%%%%%%%%%%%%%%%%%%%%%%%%%%%%%%%%%%%%%
	%%%   BIBLIOGRAPHY   %%%%%%%%%%%%%%%%%%%%
	%%%%%%%%%%%%%%%%%%%%%%%%%%%%%%%%%%%%%%%%%
	
	\pagestyle{empty}
	\rfoot{}
	\lfoot{}
	\renewcommand{\footrulewidth}{0pt}
	\printbibliography[title={\hspace{\fill}Literaturverzeichnis}]
	\thispagestyle{empty}
	
	%%%%%%%%%%%%%%%%%%%%%%%%%%%%%%%%%%%%%%%%%
	%%%   BEGLEITPROTOKOLL   %%%%%%%%%%%%%%%%
	%%%%%%%%%%%%%%%%%%%%%%%%%%%%%%%%%%%%%%%%%
	
	\includepdf[pages={1-2}]{./kapitel/anhang/Begleitprotokoll_schoergendorfer_to_show.PDF}
	\includepdf[pages={1-2}]{./kapitel/anhang/Begleitprotokoll_hoerzi_to_show.PDF}
	
	%%%%%%%%%%%%%%%%%%%%%%%%%%%%%%%%%%%%%%%%%
	%%%   LIST OF FIGURES   %%%%%%%%%%%%%%%%%
	%%%%%%%%%%%%%%%%%%%%%%%%%%%%%%%%%%%%%%%%%
	
	\rfoot{}
	\lfoot{}
	%%%%%%%%% untenstehende Gruppe ist dafür, dass keine Pagenumber beim Abbildungsverzeichnis ist
	\clearpage
	\begingroup 
	\makeatletter
	\let\ps@plain\ps@empty
	\makeatother
	
	\pagestyle{empty}
	\renewcommand{\listfigurename}{Abbildungsverzeichnis}
	\listoffigures
	\cleardoublepage
	\endgroup

	\pagebreak
	
	%%%%%%%%%%%%%%%%%%%%%%%%%%%%%%%%%%%%%%%%%
	%%%   LIST OF LISTINGS  %%%%%%%%%%%%%%%%%
	%%%%%%%%%%%%%%%%%%%%%%%%%%%%%%%%%%%%%%%%%
	
	\renewcommand{\lstlistlistingname}{Listingverzeichnis}
	\lstlistoflistings	
	\thispagestyle{empty}
	\pagebreak
	
	%%%%%%%%%%%%%%%%%%%%%%%%%%%%%%%%%%%%%%%%%
	%%%   LIST OF Tables  %%%%%%%%%%%%%%%%%%%
	%%%%%%%%%%%%%%%%%%%%%%%%%%%%%%%%%%%%%%%%%
	
	\renewcommand{\listtablename}{Tabellenverzeichnis}
	\listoftables	
	\thispagestyle{empty}
	\begin{table}[H]
		\centering
		{\rowcolors{2}{gray!20}{gray!10}
			\setlength{\arrayrulewidth}{1pt}
			\begin{tabular}[h]{|l|p{8cm}|p{2.5cm}|p{2.5cm}|}
				\hline
				\rowcolor[gray]{.6}	Abkürzung & Bedeutung \\
				\hline
				JWT & JSON Web Token\\
				\hline
				FAQ & Frequently Asked Questions \\	
				\hline
				DI & Dependency Injection \\
				\hline
				RF & Registrierungsformular \\
				\hline
				DOM & Document Object Model \\
				\hline
				API & Application Programming Interface \\
				\hline
			\end{tabular}
		}
		\caption{Abkürzungen}
	\end{table}
	
	\pagebreak
	
	%%%%%%%%%%%%%%%%%%%%%%%%%%%%%%%%%%%%%%%%%
	%%%   APPENDIX  %%%%%%%%%%%%%%%%%%%%%%%%%
	%%%%%%%%%%%%%%%%%%%%%%%%%%%%%%%%%%%%%%%%%
	
	\Appendix{
		\thispagestyle{empty}
		\begin{figure}[!hb]
			\includepdf[pages={1},  scale=.65, frame=true, pagecommand={}]{./kapitel/anhang/Pflichtenheft_ERM.pdf}
		\end{figure}
		\includepdf[pages={2-11},  scale=.65, frame=true, pagecommand={}]{./kapitel/anhang/Pflichtenheft_ERM.pdf}
	}

\end{document}